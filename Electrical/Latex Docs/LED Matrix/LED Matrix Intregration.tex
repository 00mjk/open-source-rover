\documentclass[12pt]{article}
\usepackage[margin=1in]{geometry}
\usepackage{setspace}
\usepackage{graphicx}
\usepackage{subcaption}
\usepackage{amsmath}
\usepackage{color}
\usepackage{hyperref}
\usepackage{multicol}
\usepackage{framed}
\usepackage{xcolor}
\usepackage{wrapfig}
\usepackage{float}
\usepackage{fancyhdr} 	
\usepackage{verbatim}
\pagestyle{fancy}
\lfoot{\textbf{Open Source Rover Electrical Assembly}}
\rfoot{Page \thepage}
\lhead{\textbf{\leftmark}}
\rhead{\textbf{\rightmark}}
\cfoot{}
\renewcommand{\footrulewidth}{1.8pt}
\renewcommand{\headrulewidth}{1.8pt}
\doublespacing
\setlength{\parindent}{1cm}

\begin{document}

\title{Open Source Rover}
\author{LED Matrix Integration Instructions}

\makeatletter         
\def\@maketitle{
\begin{center}	
	\makebox[\textwidth][c]{ \includegraphics[width=1.05\paperwidth]{"Pictures/LED head".png}}
	{\Huge \bfseries \sffamily \@title }\\[4ex] 
	{\huge \bfseries \sffamily \@author}\\[4ex] 
	\includegraphics[width=.85\linewidth]{"Pictures/JPL logo".png}
\end{center}}
\makeatother

\maketitle

% Introduction
\newpage


\tableofcontents

\newpage 	

\begin{figure}[H]
	\centering
	\includegraphics[width=1\textwidth]{"Pictures/parts-led matrix".PNG}
\end{figure}


\section{PCB - Logic Shifter Board} 
\subsection{Schematic/Overview}
This board is made for the mismatch between the digital logic voltage level of the raspberry pi and the LED matrix. Generally digital logic is either at 3.3V or 5V. The GPIO pins on the Raspberry Pi will output at the 3.3V logic, and the LED matrix is made to read at 5V digital logic levels. In order to have these signals make sense to each other we use a device called a logic shifter, which will take the 3.3V signals and turn them into 5V signals. 

\begin{figure}[H]
  	\centering
    	\includegraphics[width=.75\textwidth]{"Pictures/Logic Shifter Schematic".PNG}
 	\caption{Logic Shifter schematic}
	\label{ls sch}
\end{figure}

In Figure \ref{ls sch} the wire labels correspond to the name of the GPIO pin the signal comes from on the raspberry pi. The data sheet for the logic shifter IC chip can be found at: 
\begin{itemize}
	\item \href{https://cdn-shop.adafruit.com/datasheets/sn74lvc245a.pdf}{https://cdn-shop.adafruit.com/datasheets/sn74lvc245a.pdf}
\end{itemize}

\subsection{Populating the board/Testing}

The board will come initially unpopulated of components and will look like

\begin{figure}[H]
  	\centering
    	\includegraphics[width=.45\textwidth]{"Pictures/ls1".png}
 	\caption{Logic Shifter Board}
	\label{ls1}
\end{figure}

\begin{enumerate}

	\item First we want to populate the board, and test that it is working properly before fully integrating it into the rest of the system. Break off the 0.1 pitch header pins in the correct segments to fit onto the board, and insert the logic shifting chip. The chip might need the pins slightly pushed one direction in order to fit in. Solder all components into the board.

\begin{figure}[H]
 	\centering
  	\begin{minipage}[b]{0.45\textwidth}
		\includegraphics[width=\textwidth]{"Pictures/ls2".PNG}
  	\end{minipage}
  	\hfill
  	\begin{minipage}[b]{0.45\textwidth}
    		\includegraphics[width=\textwidth]{"Pictures/ls3".png}
  	\end{minipage}
	\caption{Populating the board}
	\label{ls2}
\end{figure}

	\item You need to run connections to the board now from the Raspberry Pi. Figure \ref{ls2pi} shows which pins on the board are connected to which pin on the Raspberry Pi, with the exception of the purple wire, which will be using to test the signals and drive it from the 3.3V DC power pin on the Raspberry Pi. Figure \ref{ls4} shows which of these we will need to start connecting the board to test.

\begin{figure}[H]
  	\centering
    	\includegraphics[width=.45\textwidth]{"Pictures/pins".png}
 	\caption{Board Connections}
	\label{ls4}
\end{figure}

\begin{center}
\begin{tabular}[2]{| c | c |}	
	\hline
	Red & +5V \\ \hline
	Brown & GND \\ \hline
	White & GPIO12 \\ \hline
	Grey & GPIO10 \\ \hline
	Purple & 3.3V (pin\#01) \\ \hline
\end{tabular}	
\end{center}

	\item To begin testing make sure that both chips are being powered correctly with +5V
	
\begin{figure}[H]
 	\centering
  	\begin{minipage}[b]{0.45\textwidth}
		\includegraphics[width=\textwidth]{"Pictures/ls4".PNG}
  	\end{minipage}
  	\hfill
  	\begin{minipage}[b]{0.45\textwidth}
    		\includegraphics[width=\textwidth]{"Pictures/ls5".png}
  	\end{minipage}
	\caption{Populating the board}
	\label{ls2}
\end{figure}

	\item We want to make sure we set the direction to flow from bus B (right side of chip) to bus A (left side of chip), which means that both the DIR and the OE pins should be low, 0V

\begin{figure}[H]
  	\centering
    	\includegraphics[width=.35\textwidth]{"Pictures/dir".png}
 	\caption{Chip bus direction logic}
	\label{dir}
\end{figure}

\begin{figure}[H]
 	\centering
  	\begin{minipage}[b]{0.45\textwidth}
		\includegraphics[width=\textwidth]{"Pictures/ls6".PNG}
  	\end{minipage}
  	\hfill
  	\begin{minipage}[b]{0.45\textwidth}
    		\includegraphics[width=\textwidth]{"Pictures/ls7".png}
  	\end{minipage}
	\caption{Populating the board}
	\label{ls2}
\end{figure}

	\item Now we can test the signal. Starting with the 3.3V plugged into the GPIO 2 pin (3rd pin down on the P1 set of headers) test the input pin on the chip, it should read 3.3V. Then test the output, it should read 5V.

\begin{figure}[H]
 	\centering
  	\begin{minipage}[b]{0.45\textwidth}
		\includegraphics[width=\textwidth]{"Pictures/ls8".PNG}
  	\end{minipage}
  	\hfill
  	\begin{minipage}[b]{0.45\textwidth}
    		\includegraphics[width=\textwidth]{"Pictures/ls9".png}
  	\end{minipage}
	\caption{Populating the board}
	\label{ls2}
\end{figure}

	\item Repeat this process for the rest of the pins, following along with the circuit diagram in Figure \ref{ls sch}, making sure each channel of the shifter is working correctly. 

\begin{figure}[H]
 	\centering
  	\begin{minipage}[b]{0.45\textwidth}
		\includegraphics[width=\textwidth]{"Pictures/ls10".PNG}
  	\end{minipage}
  	\hfill
  	\begin{minipage}[b]{0.45\textwidth}
    		\includegraphics[width=\textwidth]{"Pictures/ls11".png}
  	\end{minipage}
	\caption{Populating the board}
	\label{ls2}
\end{figure}

\end{enumerate}

\section{Connecting the LED Matrix}
We followed along with a project on Adafruit to hook up the Raspberry Pi to the LED matrix, with the addition of the logic shifter board inbetween the Pi and LED matrix. It can be found at:

\begin{itemize}
	\item \href{https://learn.adafruit.com/connecting-a-16x32-rgb-led-matrix-panel-to-a-raspberry-pi/overview}{https://learn.adafruit.com/connecting-a-16x32-rgb-led-matrix-panel-to-a-raspberry-pi/overview}
\end{itemize}

\noindent The Logic Shifter PCB is not included in the adafruit project, the following will be descriptions as to how to include that in the wiring and routing. Figure \ref{ls2pi} shows the pin connections on the Logic shifter, labeled with which pin they connect to on the Raspberry Pi. We suggest putting the logic shifter board in the 3D printed head, and routing the wires through the body and up through the pvp pipe "neck". 

\noindent Make sure to use the thicker gauge wire (20AWG) for powering the LED matrix, as it can pull up to 5 amps of current, we want to make sure to use wire thick enough for it, however all the other signals are digital signals which the 30 AWG wire is fine for. 

\begin{figure}[H]
 	\centering
  	\begin{minipage}[b]{0.45\textwidth}
		\includegraphics[width=\textwidth]{"Pictures/ls to pi".PNG}
  	\end{minipage}
  	\hfill
  	\begin{minipage}[b]{0.45\textwidth}
    		\includegraphics[width=\textwidth]{"Pictures/pi3_gpio".png}
  	\end{minipage}
	\caption{Raspberry pi to Logic Shifter board}
	\label{ls2pi}
\end{figure}

\textbf{**Note**} In Figure \ref{ls2pi} the pin labels on the board (left) correspond to the pin NAME not the pin number. So pin 12 is the pin NAME GPIO 12, and \textbf{Not} pin\#12 

\begin{figure}[H]
 	\centering
	\includegraphics[width=0.5\textwidth]{"Pictures/ls to led".PNG}
 	\caption{Logic Shifter to LED Matrix}
	\label{ls2led}
\end{figure}

Figure \ref{ls2led} shows the connections out of the Logic Shifter board to the LED Matrix. Make sure to use the INPUT side of the LED matrix. Labels of the pin namings can be found directly on the LED Matrix.  




\end{document}
