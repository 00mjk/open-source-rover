\documentclass[12pt]{article}
\usepackage[margin=1in]{geometry}
\usepackage{setspace}
\usepackage{graphicx}
\usepackage{subcaption}
\usepackage{amsmath}
\usepackage{color}
\usepackage{hyperref}
\usepackage{multicol}
\usepackage{framed}
\usepackage{xcolor}
\usepackage{wrapfig}
\usepackage{float}
\usepackage{fancyhdr}
\usepackage{verbatim}
\usepackage{colortbl}
\usepackage{array, booktabs, caption}
\usepackage{makecell}


\pagestyle{fancy}
\lfoot{\textbf{Open Source Rover PCB Assembly Instructions}}
\rfoot{Page \thepage}
\lhead{\textbf{\leftmark}}
\rhead{\textbf{\rightmark}}
\cfoot{}
\renewcommand{\footrulewidth}{1.8pt}
\renewcommand{\headrulewidth}{1.8pt}
\doublespacing
\setlength{\parindent}{1cm}
% Parts list tables
\renewcommand\theadfont{\bfseries}
\newcolumntype{I}{ >{\centering\arraybackslash} m{2cm} }  % part image
\newcolumntype{N}{ >{\centering\arraybackslash} m{3cm} }  % part name
\newcolumntype{Q}{ >{\centering\arraybackslash} m{0.75cm} }  % ref & qty


\begin{document}

\newcommand\partimg{\includegraphics[width=2cm,height=1.0cm,keepaspectratio]}

\title{Open Source Rover: PCB Assembly Instructions}
\author{Authors: Michael Cox, Eric Junkins}

\makeatletter
\def\@maketitle{
\begin{center}
	\makebox[\textwidth][c]{ \includegraphics[width=0.5\paperwidth]{"img/Pictures/Assembly/assembly_15".png}}
	{\Huge \bfseries \sffamily \@title }\\[3ex]
	{\Large \sffamily \@author}\\[3ex]
	\includegraphics[width=.65\linewidth]{"img/JPL logo".png}
\end{center}}
\makeatother

\maketitle

\noindent {\footnotesize Reference herein to any specific commercial product, process, or service by trade name, trademark, manufacturer, or otherwise, does not constitute or imply its endorsement by the United States Government or the Jet Propulsion Laboratory, California Institute of Technology. \textcopyright  2018 California Institute of Technology. Government sponsorship acknowledged.}


% Introduction
\newpage


\tableofcontents

\newpage

\section{PCB Assembly}
This document goes through the process of assembling and testing the custom Printed Circuit Boards for the project. One thing you might notice is the boards have reference designators on them that do not match the reference designators used in the parts lists. The board components mapping between these can be found below:

\bigskip 
\begin{frame}{}
  \centering \Huge
  \textbf{Control Board References}
\end{frame}

\begin{table}[H]
	\centering
	\begin{tabular}{| l | l | l | }
		\hline 
		\textbf{Component} & \textbf{Parts list Ref} & \textbf{Schematic Ref} \\ \hline
		Terminal block 6 pos top entry  & E4 & J1-5 \\ \hline
		Terminal block 6 pos side entry & E3 & J17-26 \\ \hline
		Term block 2p side entry (5.08mm) & E12 & J13,15,16 \\ \hline
		Connector Header pin 40P 40x1 & E15 & J8,9,11 \\ \hline
		Connector Header pin 6P 6x1 & E14 & J10 \\ \hline
		Connector Header socket 5P 5x1 & E6 & RC1-5 \\ \hline
		Connector Header socket 40P 2x20 & E13 & J6,7 \\ \hline
		Connector Header socket 20 2x10 & E5 & RC1-5 \\ \hline
		Capacitor 100nF & E11 & C1-17 \\ \hline
		Resistor 4.7K 1/4 Watt  & E7 & R1 \\ \hline
		Resistor 10K 1/4 Watt & E8 & R4,6,8,10 \\ \hline
		Resistor 22K 1/4 Watt & E9 & R3,5,7,9 \\ \hline
		Resistor 10K 1/2 Watt & E10 & R2 \\ \hline
		LM358 Op Amp & E25 & U1,2 \\ \hline
		DIP IC socket 8 Pos & E33 & U1,2 \\ \hline
		Power Diode & E17 & D1 \\ \hline
		10A Fuse & E16 & F1 \\ \hline
		USB A Connector & E34 & J12,14 \\ \hline

	\end{tabular}
\end{table}

\bigskip 
\begin{frame}{}
  \centering \Huge
  \textbf{Arduino Board References}
\end{frame}

\begin{table}[H]
	\centering
	\begin{tabular}{| l | l | l | }
		\hline 
		\textbf{Component} & \textbf{Parts list Ref} & \textbf{Schematic Ref} \\ \hline
		Term block 2P side entry & E16 & J6 \\ \hline
		Connector Header pin 40P 40x1 & E15 & J2,3,4 \\ \hline
		Connecor Header 16P 2x8 & E30 & J1 \\ \hline
		Connecor Header 6P 6x1 & E14 & J5 \\ \hline 

	\end{tabular}
\end{table}


\subsection{Control Board Assembly}

\subsubsection{Motor \& RoboClaw Connectors}

\begin{enumerate}

\begin{table}[H]
    \centering
    \arrayrulecolor{lightgray}
    \sffamily\footnotesize
    \captionsetup{font={sf,bf}}
    \caption{Parts/Tools Necessary}
    \begin{tabular}{|N|Q|Q|I|N|Q|Q|I|}
        \hline
        \thead{Item} & \thead{Ref} & \thead{Qty} & \thead{Image} & \thead{Item} & \thead{Ref} & \thead{Qty} & \thead{Image} \\ \hline
        OSR Control Board & E1 & 1 & \partimg{../../../images/components/Electronics/E1.PNG} & 6 Pos Side Term Block & E3 & 10 & \partimg{../../../images/components/Electronics/E3.PNG} \\ \hline
        6 Pos Top Term Block & E4 & 5 & \partimg{../../../images/components/Electronics/E4.PNG} & 5 Pos Header socket & E5 & 5 & \partimg{../../../images/components/Electronics/E5.PNG} \\ \hline
        5 Pos Header socket & E6 & 5 & \partimg{../../../images/components/Electronics/E6.PNG} & Soder Iron & N/A & &  \\ \hline
    \end{tabular}
\end{table}

\item Begin by soldering the 6 Position Side entry terminal blocks \textbf{E3} into the \textbf{top} side of the PCB, on the edge of the board as shown in Figure \ref{assem_1}. These terminal blocks will run motor power, encoder power, and encoder signals between the motors/encoders and the RoboClaw motor controllers. The 6 terminal blocks will each be labeled with schematic reference designators J17-J26 on the PCB. Be sure that the wire terminals face \textbf{OUTWARD} (away from the center of the board) on all of these connectors.

\begin{figure}[H]
  \centering
  \begin{minipage}[b]{0.45\textwidth}
    \includegraphics[width=\textwidth]{"img/Pictures/Assembly/assembly_1".png}
  \end{minipage}
  \hfill
  \begin{minipage}[b]{0.45\textwidth}
    \includegraphics[width=\textwidth]{"img/Pictures/Assembly/top_term".jpg}
  \end{minipage}
  \caption{Assembly Step 1}
  \label{assem_1}
\end{figure}

\item On the \textbf{bottom} of the board, solder the 6 Position top entry terminal blocks \textbf{E4}. They will be labeled with schematic reference designators J1-5. The orientation of the wire terminal face should be AWAY from the each of RoboClaw outlines (see Figure \ref{assem_2}). These terminals will run battery power and +/- motor signals to the RoboClaw motor controllers from the PCB.

\begin{figure}[H]
  \centering
  \begin{minipage}[b]{0.45\textwidth}
    \includegraphics[width=\textwidth]{"img/Pictures/Assembly/assembly_3".png}
  \end{minipage}
  \hfill
  \begin{minipage}[b]{0.45\textwidth}
    \includegraphics[width=\textwidth]{"img/Pictures/Assembly/bottom_term".jpg}
  \end{minipage}
  \caption{Assembly Step 2}
  \label{assem_2}
\end{figure}

\item On the \textbf{bottom} of the board, solder the 20-position female socket header connector \textbf{E5} and the 5-position female socket header connector \textbf{E6}. They will be labeled with reference designators RoboClaw 1-5. These are the digital signal pins for the RoboClaw motor controllers.

\begin{figure}[H]
  \centering
  \begin{minipage}[b]{0.45\textwidth}
    \includegraphics[width=\textwidth]{"img/Pictures/Assembly/assembly_5".png}
  \end{minipage}
  \hfill
  \begin{minipage}[b]{0.45\textwidth}
    \includegraphics[width=\textwidth]{"img/Pictures/Assembly/rc_header".jpg}
  \end{minipage}
  \caption{Assembly Step 3}
  \label{assem_3}
\end{figure}

\end{enumerate}

\subsubsection{Resistors and Capacitors}
\begin{table}[H]
    \centering
    \arrayrulecolor{lightgray}
    \sffamily\footnotesize
    \captionsetup{font={sf,bf}}
    \caption{Parts/Tools Necessary}
    \begin{tabular}{|N|Q|Q|I|N|Q|Q|I|}
        \hline
        \thead{Item} & \thead{Ref} & \thead{Qty} & \thead{Image} & \thead{Item} & \thead{Ref} & \thead{Qty} & \thead{Image} \\ \hline
        OSR Control Board & E1 & 1 & \partimg{../../../images/components/Electronics/E1.PNG} & 4.7K 1/4 Watt Resistor & E7 & 1 & \partimg{../../../images/components/Electronics/E7.PNG} \\ \hline
        10K 1/4 Watt Resistor & E8 & 4 & \partimg{../../../images/components/Electronics/E8.PNG} & 22K 1/4 Watt Resistor & E9 & 4 & \partimg{../../../images/components/Electronics/E9.PNG} \\ \hline
        10K 1/2 Watt Resistor & E10 & 1 & \partimg{../../../images/components/Electronics/E10.PNG} & 100nF Capacitor & E11 & 16 & \partimg{../../../images/components/Electronics/E11.PNG} \\ \hline
    \end{tabular}
\end{table}


\begin{enumerate}

\item On the \textbf{top} of the board, solder the resistors and capacitors by comparing the reference designator on the board to the part number listed in the table below. Some of these capacitors are used to store energy for powering components (to help protect against voltage fluctuations). Others are used as noise filtering mechanisms on analog signals (particularly to smooth the encoder signals). The resistors are needed to control the voltage across various components.

\begin{table}[H]
    \centering
    \arrayrulecolor{lightgray}
    \sffamily\footnotesize
    \captionsetup{font={sf,bf}}
    \caption{Resistor/Capacitor reference}
    \begin{tabular}{|c|c|c|}
        \hline
        \thead{Item} & \thead{Parts list Ref} & \thead{Schematic/Board Ref} \\ \hline
	4.7K 1/4 Watt Res & E7 & R1 \\ \hline
	10K 1/4 Watt Res & E8 & R4,6,8,10 \\ \hline
	22K 1/4 Watt Res & E9 & R3,5,7,9 \\ \hline
	10K 1/2 Watt Res & E10 & R2 \\ \hline
	100nF Cap & E11 & C1-17 \\ \hline

    \end{tabular}
\end{table}

\begin{figure}[H]
	\centering
	\includegraphics[width=0.6\textwidth]{"img/Pictures/Assembly/caps_res".jpg}
	\caption{Resistor and Capacitor soldering}
\end{figure}


\end{enumerate}

\subsubsection{Voltage Regulator connectors}
\begin{table}[H]
    \centering
    \arrayrulecolor{lightgray}
    \sffamily\footnotesize
    \captionsetup{font={sf,bf}}
    \caption{Parts/Tools Necessary}
    \begin{tabular}{|N|Q|Q|I|N|Q|Q|I|}
        \hline
        \thead{Item} & \thead{Ref} & \thead{Qty} & \thead{Image} & \thead{Item} & \thead{Ref} & \thead{Qty} & \thead{Image} \\ \hline
        OSR Control Board & E1 & 1 & \partimg{../../../images/components/Electronics/E1.PNG} & 5 Pos Header socket & E6 & 2 & \partimg{../../../images/components/Electronics/E6.PNG} \\ \hline
         & & & & Soder Iron & N/A & &  \\ \hline
    \end{tabular}
\end{table}

\begin{enumerate}

\item On the \textbf{bottom} of the board, solder the 5-position female header sockets \textbf{E6}. The sockets will have schematic reference designators of J9 and J11. These connectors are what the 12V and 5V voltage regulators will slot into.

\begin{figure}[H]
  \centering
  \begin{minipage}[b]{0.45\textwidth}
    \includegraphics[width=\textwidth]{"img/Pictures/Assembly/assembly_8".png}
  \end{minipage}
  \hfill
  \begin{minipage}[b]{0.45\textwidth}
    \includegraphics[width=\textwidth]{"img/Pictures/Assembly/regs_header".jpg}
  \end{minipage}
  \caption{Assembly Step 5}
  \label{assem_5}
\end{figure}


\end{enumerate}

\subsubsection{Power Connectors}

\begin{table}[H]
    \centering
    \arrayrulecolor{lightgray}
    \sffamily\footnotesize
    \captionsetup{font={sf,bf}}
    \caption{Parts/Tools Necessary}
    \begin{tabular}{|N|Q|Q|I|N|Q|Q|I|}
        \hline
        \thead{Item} & \thead{Ref} & \thead{Qty} & \thead{Image} & \thead{Item} & \thead{Ref} & \thead{Qty} & \thead{Image} \\ \hline
        OSR Control Board & E1 & 1 & \partimg{../../../images/components/Electronics/E1.PNG} & 2 Pos Side Terminal Block & E12 & 3 & \partimg{../../../images/components/Electronics/E12.PNG} \\ \hline
         2 Position 2.5mm Terminal Block & E18 & 1 & \partimg{../../../images/components/Electronics/E18.PNG} & Soder Iron & N/A & &  \\ \hline
    \end{tabular}
\end{table}


\begin{enumerate}

\item On the \textbf{top} of the board, solder the 2-position side entry terminal blocks \textbf{E13}. These will have schematic reference designators J14-16. Ensure that the wire terminals on these componenets face \textbf{OUTWARDS}. Also on the \textbf{top} of the board, solder the remaining 2-position 2.5mm terminal block \textbf{E18} (schematic reference designator J12). 

\begin{figure}[H]
  \centering
  \begin{minipage}[b]{0.45\textwidth}
    \includegraphics[width=\textwidth]{"img/Pictures/Assembly/assembly_9".png}
  \end{minipage}
  \hfill
  \begin{minipage}[b]{0.45\textwidth}
    \includegraphics[width=\textwidth]{"img/Pictures/Assembly/power_term".jpg}
  \end{minipage}
  \caption{Assembly Step 6}
  \label{assem_6}
\end{figure}

\end{enumerate}

\subsubsection{Op amp DIP socket}

\begin{table}[H]
    \centering
    \arrayrulecolor{lightgray}
    \sffamily\footnotesize
    \captionsetup{font={sf,bf}}
    \caption{Parts/Tools Necessary}
    \begin{tabular}{|N|Q|Q|I|N|Q|Q|I|}
        \hline
        \thead{Item} & \thead{Ref} & \thead{Qty} & \thead{Image} & \thead{Item} & \thead{Ref} & \thead{Qty} & \thead{Image} \\ \hline
        OSR Control Board & E1 & 1 & \partimg{../../../images/components/Electronics/E1.PNG} & 8 Pin DIP Socket & E33 & 2 & \partimg{../../../images/components/Electronics/E33.PNG} \\ \hline
        & & & & Soder Iron & N/A & &  \\ \hline
    \end{tabular}
\end{table}

\begin{enumerate}

\item On the \textbf{top} of the board, solder the 8 Pin DIP sockets \textbf{E33}. They will have schematic reference designators U1-2. Orientation of these sockets does not matter, but note that you WILL (later, in another document) need to plug in the Op-Amp chips themselves, and orientation of those chips is important when you eventually complete that step. 

\begin{figure}[H]
  \centering
  \begin{minipage}[b]{0.45\textwidth}
    \includegraphics[width=\textwidth]{"img/Pictures/Assembly/assembly_10".png}
  \end{minipage}
  \hfill
  \begin{minipage}[b]{0.45\textwidth}
    \includegraphics[width=\textwidth]{"img/Pictures/Assembly/dip_socket".jpg}
  \end{minipage}
  \caption{Assembly Step 7}
  \label{assem_7}
\end{figure}



\end{enumerate}

\subsubsection{RPi GPIO connector and misc headers}

\begin{table}[H]
    \centering
    \arrayrulecolor{lightgray}
    \sffamily\footnotesize
    \captionsetup{font={sf,bf}}
    \caption{Parts/Tools Necessary}
    \begin{tabular}{|N|Q|Q|I|N|Q|Q|I|}
        \hline
        \thead{Item} & \thead{Ref} & \thead{Qty} & \thead{Image} & \thead{Item} & \thead{Ref} & \thead{Qty} & \thead{Image} \\ \hline
        OSR Control Board & E1 & 1 & \partimg{../../../images/components/Electronics/E1.PNG} & 40 Pin Header connector & E13 & 2 & \partimg{../../../images/components/Electronics/E13.PNG} \\ \hline
	40 Position Header Pins & E15 & 1 & \partimg{../../../images/components/Electronics/E15.PNG} & 6 Position JST Connector & E14 & 1 & \partimg{../../../images/components/Electronics/E14.PNG} \\ \hline
         & & & & Soder Iron & N/A & &  \\ \hline
    \end{tabular}
\end{table}

\begin{enumerate}

\item On the \textbf{top} of the board, solder the 40-position header connectors \textbf{E13}. The clocking notch on the headers should face \textbf{OUTWARD} as shown in Figure \ref{assem_7}. The schematic reference designators are J6 and J7.

\begin{figure}[H]
  \centering
  \begin{minipage}[b]{0.45\textwidth}
    \includegraphics[width=\textwidth]{"img/Pictures/Assembly/assembly_11".png}
  \end{minipage}
  \hfill
  \begin{minipage}[b]{0.45\textwidth}
    \includegraphics[width=\textwidth]{"img/Pictures/Assembly/gpio".jpg}
  \end{minipage}
  \caption{Assembly Step 8}
  \label{assem_8}
\end{figure}


\item Take the 40 pin header pins \textbf{E15} and break off a 6-pin segment. On the \textbf{top} of the board, solder the 6-pin segment into schematic reference designator J8. Then, solder the JST connector \textbf{E14} into the J10 schematic reference designator. The opening in the pins on the JST connector should face \textbf{INWARD} as shown in Figure \ref{assem_9}.

\begin{figure}[H]
  \centering
  \begin{minipage}[b]{0.45\textwidth}
    \includegraphics[width=\textwidth]{"img/Pictures/Assembly/assembly_13".png}
  \end{minipage}
  \hfill
  \begin{minipage}[b]{0.45\textwidth}
    \includegraphics[width=\textwidth]{"img/Pictures/Assembly/jst_top".jpg}
  \end{minipage}
  \caption{Assembly Step 9}
  \label{assem_9}
\end{figure}


\end{enumerate}

\subsubsection{Fuse and Diode}

\begin{table}[H]
    \centering
    \arrayrulecolor{lightgray}
    \sffamily\footnotesize
    \captionsetup{font={sf,bf}}
    \caption{Parts/Tools Necessary}
    \begin{tabular}{|N|Q|Q|I|N|Q|Q|I|}
        \hline
        \thead{Item} & \thead{Ref} & \thead{Qty} & \thead{Image} & \thead{Item} & \thead{Ref} & \thead{Qty} & \thead{Image} \\ \hline
        OSR Control Board & E1 & 1 & \partimg{../../../images/components/Electronics/E1.PNG} & 10 Amp Fuse & E16 & 1 & \partimg{../../../images/components/Electronics/E16.PNG} \\ \hline
        Diode & E17 & 1 & \partimg{../../../images/components/Electronics/E17.PNG} & Soder Iron & N/A & &  \\ \hline
    \end{tabular}
\end{table}

\begin{enumerate}

\item Solder the fuse \textbf{E16} and diode \textbf{E17} onto the top of the board into schematic reference designators F1 (fuse) and D1 (diode). \textbf{Pay careful attention to the direction that the Diode is mounted on the board}, as it will only work in one direction!

\begin{figure}[H]
  \centering
  \begin{minipage}[b]{0.45\textwidth}
    \includegraphics[width=\textwidth]{"img/Pictures/Assembly/fuse_diode".PNG}
  \end{minipage}
  \hfill
  \begin{minipage}[b]{0.45\textwidth}
    \includegraphics[width=\textwidth]{"img/Pictures/Assembly/fuse".jpg}
  \end{minipage}
  \caption{Fuse and Diode Soldering}
  \label{assem_10}
\end{figure}

\end{enumerate}



\subsubsection{USB connectors}

\begin{table}[H]
    \centering
    \arrayrulecolor{lightgray}
    \sffamily\footnotesize
    \captionsetup{font={sf,bf}}
    \caption{Parts/Tools Necessary}
    \begin{tabular}{|N|Q|Q|I|N|Q|Q|I|}
        \hline
        \thead{Item} & \thead{Ref} & \thead{Qty} & \thead{Image} & \thead{Item} & \thead{Ref} & \thead{Qty} & \thead{Image} \\ \hline
        OSR Control Board & E1 & 1 & \partimg{../../../images/components/Electronics/E1.PNG} & USB Connector & E34 & 2 & \partimg{../../../images/components/Electronics/E34.PNG} \\ \hline
         & & & & Soder Iron & N/A & &  \\ \hline
    \end{tabular}
\end{table}

\begin{enumerate}

\item On the \textbf{top} of the board, solder the two USB Connectors \textbf{E34}. They will have reference designators J12 and J14.

\begin{figure}[H]
  \centering
  \begin{minipage}[b]{0.45\textwidth}
    \includegraphics[width=\textwidth]{"img/Pictures/Assembly/assembly_14".png}
  \end{minipage}
  \hfill
  \begin{minipage}[b]{0.45\textwidth}
    \includegraphics[width=\textwidth]{"img/Pictures/Assembly/usb".jpg}
  \end{minipage}
  \caption{Assembly Step 10}
  \label{assem_10}
\end{figure}

\end{enumerate}

\subsubsection{Standoffs}

\begin{table}[H]
    \centering
    \arrayrulecolor{lightgray}
    \sffamily\footnotesize
    \captionsetup{font={sf,bf}}
    \caption{Parts/Tools Necessary}
    \begin{tabular}{|N|Q|Q|I|N|Q|Q|I|}
        \hline
        \thead{Item} & \thead{Ref} & \thead{Qty} & \thead{Image} & \thead{Item} & \thead{Ref} & \thead{Qty} & \thead{Image} \\ \hline
        \#6-32 x 3/4" Threaded Standoff & T3 & 4 & \partimg{../../../images/components/Standoffs/T3.png} & \#4-40 x 1/2 " Threaded Standoff & T11 & 20 & \partimg{../../../images/components/Standoffs/T11.png} \\ \hline
        \#2-56 Threaded Standoff & T6 & 4 & \partimg{../../../images/components/Standoffs/T6.png} & M2.5 x 10mm Threaded Standoff & T10 & 4 & \partimg{../../../images/components/Standoffs/T11.png} \\ \hline
        \#6-32 3/8" Button Head Screw & B2 & 8 & \partimg{../../../images/components/Screws/B2.png} & \#4-40 1/4" Button head Screw & B8 & 40 & \partimg{../../../images/components/Screws/B8.png} \\ \hline
        \#2-56 1/4" Button head Screw & B13 & 8 & \partimg{../../../images/components/Screws/B13.png} & M2.5 x 6mm & B10 & 8 & \partimg{../../../images/components/Screws/B10.png} \\ \hline
    \end{tabular}
\end{table}

\begin{enumerate}
\item \textbf{Board mounting Standoffs:} On the \textbf{BOTTOM} side of the board, attach the four \#6-32 Standoffs \textbf{T4} on the outermost corner four mounting holes using screws \textbf{B2}.  These standoffs will be used to attach the board to the robot chassis.

\begin{figure}[H]
  \centering
  \begin{minipage}[b]{0.45\textwidth}
    \includegraphics[width=\textwidth]{"img/Pictures/Assembly/standoff_2".png}
  \end{minipage}
  \hfill
  \begin{minipage}[b]{0.45\textwidth}
    \includegraphics[width=\textwidth]{"img/Pictures/Assembly/standoff_1".png}
  \end{minipage}
  \caption{Mounting Standoffs}
  \label{standoffs_1}
\end{figure}

\item \textbf{RoboClaw mounting Standoffs:} On the \textbf{BOTTOM} side of the board, attach the twenty \#4-40 Standoffs \textbf{T5} using screws \textbf{B8} as shown in Figure \ref{standoffs_2}. You can identify the RoboClaw mounting holes as the ones that are inside the RoboClaw rectangles on the silk screen.


\begin{figure}[H]
	\centering
	\includegraphics[width=0.6\textwidth]{"img/Pictures/Assembly/standoff_3".PNG}
  \caption{RoboClaw Mounting Standoffs}
  \label{standoffs_2}
\end{figure}

\item \textbf{Voltage Regulator Standoffs:} On the \textbf{BOTTOM} side of the board, attach the four \#2-56 Standoffs \textbf{T6} using screws \textbf{B13} as shown in Figure \ref{standoffs_3}. 

\begin{figure}[H]
	\centering
	\includegraphics[width=0.6\textwidth]{"img/Pictures/Assembly/standoff_4".PNG}
  \caption{Voltage regulator Mounting Standoffs}
  \label{standoffs_3}
\end{figure}

\item \textbf{Raspberry Pi  Standoffs:} On the \textbf{TOP} side of the board, attach the four M2.5 Standoffs \textbf{T7} using screws \textbf{B10} as shown in Figure \ref{standoffs_4}. 

\begin{figure}[H]
	\centering
	\includegraphics[width=0.6\textwidth]{"img/Pictures/Assembly/standoff_5".PNG}
  \caption{Raspberry Pi Standoffs}
  \label{standoffs_4}
\end{figure}

\begin{figure}[H]
	\centering
	\includegraphics[width=0.6\textwidth]{"img/Pictures/Assembly/standoffs".jpg}
	\caption{Standoffs Installed (Bottom View)}
\end{figure}


\end{enumerate}


\subsection{Arduino Sheild Assembly}

\begin{table}[H]
    \centering
    \arrayrulecolor{lightgray}
    \sffamily\footnotesize
    \captionsetup{font={sf,bf}}
    \caption{Parts/Tools Necessary}
    \begin{tabular}{|N|Q|Q|I|N|Q|Q|I|}
        \hline
        \thead{Item} & \thead{Ref} & \thead{Qty} & \thead{Image} & \thead{Item} & \thead{Ref} & \thead{Qty} & \thead{Image} \\ \hline
        Arduino Sheild & E2 & 1 & \partimg{../../../images/components/Electronics/E2.PNG} & 2 Position Term Block & E18 & 1 & \partimg{../../../images/components/Electronics/E18.PNG} \\ \hline
         2x8 Shrouded Header pins & E19 &  1 & \partimg{../../../images/components/Electronics/E19.PNG} & 1x6 JST header pins & E14 & 1 & \partimg{../../../images/components/Electronics/E14.PNG} \\ \hline
	1x40 0.1 Pitch Header pins & E15 & 1 & \partimg{../../../images/components/Electronics/E15.PNG} & & & & \\ \hline
    \end{tabular}
\end{table}

Now, we will assemble the Arduino shield that will sit in the rover's head and control the face.  We will start with the \textbf{TOP} side of the Arduino board. The final product is shown in Figure \ref{assembled_arduino_sheild_top}.

\begin{figure}[H]
  \centering
  \begin{minipage}[b]{0.6\textwidth}
    \includegraphics[width=\textwidth]{"img/arduino_sheild_top".jpg}
  \end{minipage}
  \caption{Assembled Arduino Sheild}
  \label{assembled_arduino_sheild_top}
\end{figure}


\begin{enumerate}

	\item Begin by taking the 2-position terminal block \textbf{E18} and soldering it to top of the board at the J6 connector, such that the screw terminals face outwards as shown in Figure \ref{assembled_arduino_sheild_top}:


\begin{figure}[H]
	\centering
	\includegraphics[width=0.75\textwidth]{"img/term_block".png}
	\caption{Terminal Block assembly}
\end{figure}


	\item Solder the 1x6 JST connector \textbf{E14} onto the top of the board at the J5 connector such that the notch in the connector faces \textbf{OUTWARDS} as shown in Figure \ref{assembled_arduino_sheild_top}:

\begin{figure}[H]
	\centering
	\includegraphics[width=0.75\textwidth]{"img/jst".png}
	\caption{JST assembly}
\end{figure}

	\item Solder the 2x8 shrouded header pin connector \textbf{E19} to the top of the board at connector J1 such that the notch faces \textbf{OUTWARDS} as shown in Figure \ref{assembled_arduino_sheild_top}.

\begin{figure}[H]
	\centering
	\includegraphics[width=0.75\textwidth]{"img/2x8_above".png}
	\caption{Terminal Block assembly}
\end{figure}

	\item Solder the 0.1 pitch header pins \textbf{E15} to connectors J2-4 on the top side of the board:

\begin{figure}[H]
	\centering
	\includegraphics[width=0.75\textwidth]{"img/01_pitch_above".png}
	\caption{0.1 Pitch headers}
\end{figure}

Flip the Arduino board over to the \textbf{BOTTOM} side where we will now install the remaining headers. The final product is shown in Figure \ref{assembled_arduino_sheild_bottom}.

	\item Solder the 0.1 pitch header pins \textbf{E15} to the bottom side of the board in the remaining hole sets.

\begin{figure}[H]
	\centering
	\includegraphics[width=0.75\textwidth]{"img/01_pitch_under".png}
	\caption{0.1 Pitch headers}
\end{figure}

\begin{figure}[H]
  \centering
  \begin{minipage}[b]{0.75\textwidth}
    \includegraphics[width=\textwidth]{"img/arduino_sheild_Bottom".jpg}
  \end{minipage}
  \caption{Assembled Arduino Sheild}
  \label{assembled_arduino_sheild_bottom}
\end{figure}

\end{enumerate}

\section{Component Integration and Testing}

This next section will go over the process of integrating the electronics onto the Control Board and the testing to verify that the board and components are working as expected at each step. You should perform this section with the board \textbf{outside} of the robot chassis in case you need to replace components or fix any mistakes. It is important to do the following steps \textbf{one at time} so you can verify that electronics are working as intended. These tests will save you from accidentally breaking one or more of your components if something else is plugged in incorrectly or shorted.

\subsection{Testing the Control Board}
\subsubsection{Power Distribution System}

\begin{table}[H]
    \centering
    \arrayrulecolor{lightgray}
    \sffamily\footnotesize
    \captionsetup{font={sf,bf}}
    \caption{Parts/Tools Necessary}
    \begin{tabular}{|N|Q|Q|I|N|Q|Q|I|}
        \hline
        \thead{Item} & \thead{Ref} & \thead{Qty} & \thead{Image} & \thead{Item} & \thead{Ref} & \thead{Qty} & \thead{Image} \\ \hline
        OSR Control Board & E1 & 1 & \partimg{../../../images/components/Electronics/E1.PNG} & Battery & E36 & 1 & \partimg{../../../images/components/Electronics/E36.PNG} \\ \hline
         Tamiya Battery Connectors & E35 & 1 & \partimg{../../../images/components/Electronics/E35.PNG} & Red 20 AWG & X1 & 1 & \partimg{../../../images/components/Wiring/X1.PNG} \\ \hline
	Black 20 AWG & X2 & 1 & \partimg{../../../images/components/Wiring/X2.PNG} & 5V Regulator & E22 & 1 & \partimg{../../../images/components/Electronics/E22.PNG} \\ \hline
	12V Regulator & E23 & 1 & \partimg{../../../images/components/Electronics/E23.PNG} & RoboClaw Motor Controller & E20 & 5 & \partimg{../../../images/components/Electronics/E20.PNG} \\ \hline
	Op-Amp LM358P & E25 & 2 & \partimg{../../../images/components/Electronics/E25.PNG} & Micro USB Cable & E28 & 1 & \partimg{../../../images/components/Electronics/E28.PNG} \\ \hline
    \end{tabular}
\end{table}

\begin{enumerate}

\item Begin by powering the board. For testing purposes, we will plug the battery directly into the board, bypassing the switch and volt meter.  Thus, the connection we use for testing will look a little bit different than when you fully install the board into the rover. Insert the red wire on the Tamiya Battery Connector \textbf{E35} into the IN terminal on connector \textbf{J16 / POWER SWITCH}, and the black wire to the GND terminal on the connector \textbf{J15 / BATTERY IN}.

\begin{figure}[H]
  \centering
  \begin{minipage}[b]{0.45\textwidth}
    \includegraphics[width=\textwidth]{"img/Pictures/Testing/testing_1".png}
  \end{minipage}
  \hfill
  \begin{minipage}[b]{0.45\textwidth}
    \includegraphics[width=\textwidth]{"img/Pictures/Testing/testing_2".PNG}
  \end{minipage}
  \caption{Test Step 1}
  \label{test_1}
\end{figure}

\item Using a Digital Multimeter (DMM), probe the voltage across the test points T1 and T2. These will tell you the voltage at which the board power rails are, which should be the direct voltage of the battery. Verify that from T1 to T2 reads a positive number, and is between 12V and 16.7V depending on the charge state of your battery. 

\begin{figure}[H]
  \centering
    \includegraphics[width=.85\textwidth]{"img/Pictures/Testing/testing_3".PNG}
  \caption{Test pads}
  \label{test_pads_1}
\end{figure}

\item Next, you will need to create jumper wires that will connect the RoboClaw motor controller power terminals to the RoboClaws. \textbf{Unplug the Tamiya battery connector before plugging in or unplugging ANY components, or before inserting components!!!}  (in future steps, we will not explicity say to unplug the battery, but \textbf{you should disconnect the battery at each step BEFORE inserting components or working on the board!}). Take the red and black 24 AWG wires \textbf{W1 and W2} and cut 15 2-inch long segments of each color (you should have 15 red and 15 black pieces). Using wire strippers, strip the ends about 0.1 inches at each end. Insert these jumper wires into the terminal blocks on the RoboClaw Motor Controllers \textbf{E20} in the following way:

\begin{table}[H]
    \centering
    \arrayrulecolor{lightgray}
    \sffamily\footnotesize
    \captionsetup{font={sf,bf}}
    \caption{Parts/Tools Necessary}
	\begin{tabular}{| c|c|}
		\hline
		\thead{Terminal} & \thead{Wire Color} \\ \hline
		M1A & Red \\ \hline
		M1B & Black \\ \hline
		+ & Red \\ \hline
		- & Black \\ \hline
		M2A & Red \\ \hline
		M2B & Black \\ \hline
	\end{tabular}
\end{table}

\item Start by inserting one of the RoboClaws into the slot on the bottom of the board labeled ROBOCLAW 2. Connect the wires directly across from RoboClaw motor terminal block to the terminal block on the control board as shown in Figure \ref{roboclaw_power_wires}. 

\begin{figure}[H]
  \centering
  \begin{minipage}[b]{0.40\textwidth}
    \includegraphics[width=\textwidth]{"img/Pictures/Assembly/rc_wires_2".jpg}
  \end{minipage}
  \hfill
  \begin{minipage}[b]{0.55\textwidth}
    \includegraphics[width=\textwidth]{"img/Pictures/Assembly/rc_wires".jpg}
  \end{minipage}
  \label{roboclaw_power_wires}
  \caption{RoboClaw power/motor wires}
\end{figure}


\item Plug in the battery. An LED on the RoboClaw will turn on; verify that it is green. If the LED is red, it means there is an error. Error codes can be traced by looking at the RoboClaw user manual:

\begin{itemize}
	\item \href{https://www.basicmicro.com/downloads}{https://www.basicmicro.com/downloads}
\end{itemize}

\item Repeat this process one RoboClaw at a time \textbf{following the order of 2, 3, 4, 1, 5} (testing each RoboClaw after you plug it in) until all 5 RoboClaws have been plugged into the board.

\item Take the two voltage regulators \textbf{E23 and 24} and solder their header pins to the bottom side of the board.  Note that you will be soldering the back side of the pins on the top side of the board (the side with large capacitors on it). 

\item Insert the 5V regulator into the control board as shown in Figure \ref{test_5}. Power your board and probe between test points T4 and T2 on the top of the board (Figure \ref{test_5}) with your DMM and verify that it reads 5V. If it does not, make sure that the 5V regulator is slotted in properly and that your solder connections are solid.

\begin{figure}[H]
  \centering
  \begin{minipage}[b]{0.45\textwidth}
    \includegraphics[width=\textwidth]{"img/Pictures/Testing/testing_7".png}
  \end{minipage}
  \hfill
  \begin{minipage}[b]{0.45\textwidth}
    \includegraphics[width=\textwidth]{"img/Pictures/Testing/testing_3".png}
  \end{minipage}
  \caption{5V Regulator \& Test Pads}
  \label{test_5}
\end{figure}

\item Insert the 12V regulator into the control board. Power your board and probe between test point T5 and T6 on the top of the board with your DMM and verify that it reads 12V. If it does not, make sure the 12V regulator is slotted in properly and that your solder connections are solid.

\begin{figure}[H]
  \centering
  \begin{minipage}[b]{0.45\textwidth}
    \includegraphics[width=\textwidth]{"img/Pictures/Testing/testing_8".png}
  \end{minipage}
  \hfill
  \begin{minipage}[b]{0.45\textwidth}
    \includegraphics[width=\textwidth]{"img/Pictures/Testing/test_pads_t5_t6".jpg}
  \end{minipage}
  \caption{12V Regulator \& Test Pads}
  \label{test_6}
\end{figure}

If all voltage test points read expected values and all the RoboClaw motor Controllers have green LEDs, the power system has been verified! 

\end{enumerate}

\subsection{Op-Amp Integration}

Press the Op-Amp LM358P \textbf{E25} into the slots in the 8 Position DIP socket. Take careful note of the direction of the chip in the DIP socket, as the notch \textbf{MUST} face the correct direction.

\begin{figure}[H]
  \centering
    \includegraphics[width=.85\textwidth]{"img/Pictures/Assembly/op_amp".PNG}
  \caption{Op-amp integration}
\end{figure}

\subsubsection{Voltage Divider Verification}

Now that the Op-amps are installed we want to check and make sure the voltage dividers are working correctly. To do this, we will run power from the motors directly back into the analog read signal, and see what voltage it gets divided down to. On the motor connectors for each corner motor (J23-26), use a jumper wire to connect the 5V signal line to the ENx line. Then, use a DMM to measure the voltage between each of the following test pads and GND and compare the values to the expected voltage ranges in Table \ref{voltage_divider_table}:

\begin{table}[H]
    \centering
    \arrayrulecolor{lightgray}
    \sffamily\footnotesize
    \captionsetup{font={sf,bf}}
    \caption{Parts/Tools Necessary}
	\begin{tabular}{| l | l | l |}
		\hline
		\thead{Signal} & \thead{Test Pad} & \thead{Voltage (to ground, in Volts)} \\ \hline
		OP amp Power                  & T13   & 5 \\ \hline
		M7 Encoder signal divided  & T7  & 1.5 - 2.0 \\ \hline
		M7 Encoder raw                & T9  & 5 \\ \hline
		M7 Encoder after Op-amp  & T11 & 1.5 - 2.0 \\ \hline
		M8 Encoder signal divided  & T18  & 1.5 - 2.0 \\ \hline
		M8  Encoder raw               & T20  & 5 \\ \hline
		M8 Encoder after Op-amp  & T16 & 1.5 - 2.0 \\ \hline
		M9 Encoder signal divided  & T8  & 1.5 - 2.0 \\ \hline
		M9  Encoder raw               & T10  & 5 \\ \hline
		M9 Encoder after Op-amp  & T12 & 1.5 - 2.0 \\ \hline
		M10 Encoder signal divided & T19 & 1.5 - 2.0 \\ \hline
		M10  Encoder raw              & T21  & 5 \\ \hline
		M10 Encoder after Op-amp  & T17 & 1.5 - 2.0 \\ \hline
  \end{tabular}
  \label{voltage_divider_table}
\end{table}

\subsubsection{RoboClaw Testing and Verification}

In this section you will be going one by and and testing the operation of the RoboClaw Motor controllers. You will be doing this by using the GUI provided by the manufacturer of the motor contollers. The GUI can be found at the following link, under general downloads, then BasicMicro Motion Studio

\begin{itemize}
	\item \href{https://www.basicmicro.com/downloads}{https://www.basicmicro.com/downloads}
\end{itemize}


\noindent To use the GUI, insert a USB to micro USB cable from your computer to the motor controller you are going to be testing.

\noindent You must now make a temporary connection between the motor controllers and your motors.  We found it easiest to test using a set of male-male jumper wires connected between the motor terminal being tested and a test motor.

\subsubsection{Drive Motor Blocks}

\noindent Do each of the steps below for the terminal blocks labeled J17-22; these terminal blocks correspond to the driving motors for the rover. \textbf{Make sure that while you are plugging in connections, your board is powered off!}  The terminal blocks correspond to the motor controller outputs in the following manner:

\begin{table}[H]
    \centering
    \arrayrulecolor{lightgray}
    \sffamily\footnotesize
    \captionsetup{font={sf,bf}}
    \caption{Parts/Tools Necessary}
	\begin{tabular}{| l | l | l |}
		\hline
		\thead{Terminal Block Label} & \thead{RoboClaw Board Label} & \thead{Motor Output Channel} \\ \hline
		J17   & RC1  & M1 \\ \hline
		J18   & RC1  & M2 \\ \hline
		J19   & RC2  & M1 \\ \hline
		J20   & RC2  & M2 \\ \hline
		J21   & RC3  & M1 \\ \hline
		J22   & RC3  & M2 \\ \hline
	\end{tabular}
\end{table}

\begin{enumerate}

\item First, connect the wires in the following manner:

\begin{table}[H]
    \centering
    \arrayrulecolor{lightgray}
    \sffamily\footnotesize
    \captionsetup{font={sf,bf}}
    \caption{Parts/Tools Necessary}
	\begin{tabular}{| l | l | l |}
		\hline
		\thead{Signal} & \thead{Terminal Block Label} & \thead{Motor Connector Wire Color} \\ \hline
		Motor (+)  & M+  & Red \\ \hline
		Motor (-)   & M-   & Black \\ \hline
		Ground     & GND & Green \\ \hline
		+5V         & +5V  & Blue \\ \hline
		Encoder A & ENA  & Yellow \\ \hline
		Encoder B & ENB   & White \\ \hline
	\end{tabular}
\end{table}

\item Power on the board. After a minute or so, in the Basic Motion GUI you should see an available device appear. It might require an update to proceed; install the latest firmware update and then connect to the device.

\item Click on the PWM tab. We will now send a PWM signal to the motor and test that connections are all made correctly to the motor and encoder. 

\item Slowly move the slide bar for the corresponding motor output channel (Either M1 or M2 from the above table) for the terminal you are testing. Verify that the motor spins (we will worry about direction later), and that the encoder value is also changing (we'll worry about it increasing or decreasing correctly later as well). Switch direction of the slide bar and verify that it spins the other direction and the encoder value does the opposite of previous as well. If these are not happening or are backwards, go back and check that you are using the correct motor controller, terminal block, etc. If all your connections are correct, you may have to test your solder contact between the components on the board itself.

\item Repeat the steps above for each of the drive motor terminal blocks, labeled J17-22.

\end{enumerate}

\subsubsection{Corner Motor Blocks}

Do the following procedure for the terminal blocks labeled J23-26. These correspond to the corner motors for the rover. Terminal blocks correspond to the motor controller outputs in the following manner:

\begin{table}[H]
    \centering
    \arrayrulecolor{lightgray}
    \sffamily\footnotesize
    \captionsetup{font={sf,bf}}
    \caption{Parts/Tools Necessary}
	\begin{tabular}{| l | l | l |}
		\hline
		\thead{Terminal Block Label} & \thead{RoboClaw Board Label} & \thead{Motor Output Channel} \\ \hline
		J23   & RC4  & M1 \\ \hline
		J24   & RC4  & M2 \\ \hline
		J25   & RC5  & M1 \\ \hline
		J26   & RC5  & M2 \\ \hline
	\end{tabular}
\end{table}

\begin{enumerate}

\item Connect the wires to the motor in the following manner

\begin{table}[H]
    \centering
    \arrayrulecolor{lightgray}
    \sffamily\footnotesize
    \captionsetup{font={sf,bf}}
    \caption{Parts/Tools Necessary}
	\begin{tabular}{| l | l | l |}
		\hline
		\thead{Signal} & \thead{Terminal Block Label} & \thead{Motor Connector Wire Color} \\ \hline
		Motor (+)  & M+  & Red \\ \hline
		Motor (-)   & M-   & Black \\ \hline
	\end{tabular}
\end{table}

\item The main difference between the drive and corner motor systems is that for the corner system, we need to use the encoders. We want to test the voltage division circuit used on the control board; this divider which will expects a 0-5V signal from the absolute Hall effect encoder. To simulate the encoder, connect the +5V terminal on the motor terminal block straight into the ENA signal in the same terminal block.

\item Connect to the motor controller in the Basic Motion GUI. Under the General settings tab (under Encoders), change the type of encoder from Quadrature to Absolute. You should see that the encoder values change to a number somewhere around 1600. As long as it is a fairly constant value and is in the range of 1400-2000 then everything is working. If the value varies wildly or is not in the 1400-2000 range, recheck that the OP-amp is installed in the correct direction. If this number still isn't correct then make sure you correctly installed all the resistors/capacitors in the assembly steps.

\item Under the PWM tab, move the slide bar and verify that the motor spins accordingly.

\item Repeat this process for all the corner motor terminal blocks, labeled J23-26.

\end{enumerate}

\subsubsection{Raspberry Pi Install}

Next up is to verify that power to the Raspberry Pi is working. For this, you'll need a working operating system installed on the SD card. Take a moment now to follow to the Software Install steps to install the rover software on the Raspberry Pi.

\href{https://github.com/nasa-jpl/open-source-rover/blob/master/Software/Software%20Steps.pdf}{https://github.com/nasa-jpl/open-source-rover/blob/master/Software/Software\%20Steps.pdf}


\begin{table}[H]
    \centering
    \arrayrulecolor{lightgray}
    \sffamily\footnotesize
    \captionsetup{font={sf,bf}}
    \caption{Parts/Tools Necessary}
    \begin{tabular}{|N|Q|Q|I|N|Q|Q|I|}
        \hline
        \thead{Item} & \thead{Ref} & \thead{Qty} & \thead{Image} & \thead{Item} & \thead{Ref} & \thead{Qty} & \thead{Image} \\ \hline
        OSR Control Board & E1 & 1 & \partimg{../../../images/components/Electronics/E1.PNG} & Raspbery Pi 3B & E21 & 1 & \partimg{../../../images/components/Electronics/E21.PNG} \\ \hline
         40 Pin Ribbon Cable & E29 & 1 & \partimg{../../../images/components/Electronics/E29.PNG} & USB to Micro USB Cable& E27 & 1 & \partimg{../../../images/components/Electronics/E27.PNG} \\ \hline
    \end{tabular}
\end{table}

\noindent \textbf{ONLY PROCEED WITH THE FOLLOWING ONCE YOU HAVE SUCCESSFULLY FINISHED THE INSTALLATION OF THE ROVER CODE ON YOUR RASPBERRY PI.}

\begin{enumerate}

\item Attach the Raspberry Pi to the board on top of the standoffs you attached earlier, making sure that the USB ports face downward on the board as shown in Figure \ref{rpi_install}.

\item Plug in the micro USB cable \textbf{E27} to the USB power port labeled J12 and to the \textbf{power port} of the Raspberry Pi (labelled 'PWR'). Then, plug in the ribbon Cable \textbf{E29} into the Raspberry Pi GPIO header pins and connect the other end to the \textbf{J6} 40 pin GPIO connector. 

\begin{figure}[H]
  \centering
  \begin{minipage}[b]{0.55\textwidth}
    \includegraphics[width=\textwidth]{"img/Pictures/Assembly/rpi_power".jpg}
  \end{minipage}
  \hfill
  \begin{minipage}[b]{0.35\textwidth}
    \includegraphics[width=\textwidth]{"img/Pictures/Assembly/rpi_gpio".PNG}
  \end{minipage}
  \caption{RPi Install}
  \label{rpi_install}
\end{figure}


\end{enumerate}

\subsection{Arduino Sheild Testing}

\begin{table}[H]
    \centering
    \arrayrulecolor{lightgray}
    \sffamily\footnotesize
    \captionsetup{font={sf,bf}}
    \caption{Parts/Tools Necessary}
    \begin{tabular}{|N|Q|Q|I|N|Q|Q|I|}
        \hline
        \thead{Item} & \thead{Ref} & \thead{Qty} & \thead{Image} & \thead{Item} & \thead{Ref} & \thead{Qty} & \thead{Image} \\ \hline
        Arduino Sheild & E2 & 1 & \partimg{../../../images/components/Electronics/E2.PNG} & Arduino Uno & E24 & 1 & \partimg{../../../images/components/Electronics/E24.PNG} \\ \hline
	1x6 JST Cable & E26 & 1 & \partimg{../../../images/components/Electronics/E26.PNG} & 16x32 LED Matrix & E37 & 1 & \partimg{../../../images/components/Electronics/E37.PNG} \\ \hline
    \end{tabular}
\end{table}

Note: Testing the arduino board is dependent on finishing the control board and having it tested fully. Do not proceed until you have successfully tested your main control board.

Slot the Arduino Uno onto the bottom of the Arduino shield, matching the footprints on the board.

\begin{figure}[H]
  \centering
    \includegraphics[width=.85\textwidth]{"img/Pictures/Assembly/arduino_mounted".jpg}
  \caption{Arduino Sheild Mounted}
\end{figure}


\begin{enumerate}

	\item Plug in the 1x6 JST cable \textbf{E26} into the Arduino shield and to the Control board. This cable will run 12V, 5V, GND, and two serial communication lines from the main rover to the Arduino shield which runs the screen in the head. This step relies on successful tests of the 5V and 12V regulators on the control board and verifying that they work correctly.  If you have not tested the main board successfully, you may damage the Arduino.

	\item Using a Digital Multimeter, probe the following \textbf{Arduino shield test pads} and verify their voltages:
		\begin{itemize}
			\item TP1 to TP6 should read +12V. This voltage powers the Arduino Uno board
			\item TP5 to TP6 should read +5V. This voltage powers the LED Matrix and runs the LEDs on it
			\item TP4 to TP2 should read +5V. This is the 5V converter on the Arduino board
			\item TP3 to TP2 should read +3.3V. This is the 3.3V converter on the Arduino board
				
		\end{itemize}

\end{enumerate}

If all the above test points read the correct voltages, then the Arduino Sheild board is working correctly!  You are now ready to finish the Electrical Assembly of the rover!

\end{document}