\documentclass[12pt]{article}
\usepackage[margin=1in]{geometry}
\usepackage{setspace}
\usepackage{graphicx}
\usepackage{subcaption}
\usepackage{amsmath}
\usepackage{color}
\usepackage{hyperref}
\usepackage{multicol}
\usepackage{framed}
\usepackage{xcolor}
\usepackage{wrapfig}
\usepackage{float}
\usepackage{fancyhdr}
\usepackage{verbatim}
\usepackage{colortbl}
\usepackage{array, booktabs, caption}
\usepackage{makecell}

\pagestyle{fancy}
\lfoot{\textbf{Open Source Rover Mechanical Assembly Manual}}
\rfoot{Page \thepage}
\lhead{\textbf{\leftmark}}
\rhead{\textbf{\rightmark}}
\cfoot{}
\renewcommand{\footrulewidth}{1.8pt}
\renewcommand{\headrulewidth}{1.8pt}
\doublespacing
\setlength{\parindent}{1cm}

% Parts list tables
\renewcommand\theadfont{\bfseries}
\newcolumntype{I}{ >{\centering\arraybackslash} m{2cm} }  % part image
\newcolumntype{N}{ >{\centering\arraybackslash} m{3cm} }  % part name
\newcolumntype{Q}{ >{\centering\arraybackslash} m{0.75cm} }  % ref & qty


\begin{document}

\newcommand\partimg{\includegraphics[width=2cm,height=1.25cm,keepaspectratio]}


\title{Open Source Rover: Rocker-Bogie Assembly Instructions}
\author{Authors: Michael Cox, Eric Junkins Olivia Lofaro}

\makeatletter
\def\@maketitle{
\begin{center}
	\makebox[\textwidth][c]{ \includegraphics[width=0.75\paperwidth]{"Pictures/Rocker-Bogie/Rocker Bogie Title".png}}
	{\Huge \bfseries \sffamily \@title }\\[3ex]
	{\Large \sffamily \@author}\\[3ex]
	\includegraphics[width=.65\linewidth]{"Pictures/Misc/JPL logo".png}
\end{center}}
\makeatother

\maketitle

\noindent {\footnotesize Reference herein to any specific commercial product, process, or service by trade name, trademark, manufacturer, or otherwise, does not constitute or imply its endorsement by the United States Government or the Jet Propulsion Laboratory, California Institute of Technology. \textcopyright  2018 California Institute of Technology. Government sponsorship acknowledged.}

% Introduction

\newpage

\tableofcontents

\newpage
\section{Machining/Fabrication Steps}


\begin{table}[H]
    \centering
    \arrayrulecolor{lightgray}
    \sffamily\footnotesize
    \captionsetup{font={sf,bf}}
    \caption{Parts/Tools Necessary}
    \begin{tabular}{|N|Q|Q|I|N|Q|Q|I|}
        \hline
        \thead{Item} & \thead{Ref} & \thead{Qty} & \thead{Image} & \thead{Item} & \thead{Ref} & \thead{Qty} & \thead{Image} \\
        \hline
        3" Channel & S2 & 4 & \partimg{../../../images/parts_list/S2.jpg} & Metal Hacksaw or Bandsaw & & & \partimg{../../../images/parts_list/D4.png} \\ \hline
    \end{tabular}
\end{table}

\subsection{Cutting the Aluminum Channels}
On piece \textbf{S2}, measure the distance shown in Figure \ref {channel dimensions} from the edge and mark a straight line from the opposite corner to that point. Cut off the corner along that line\footnote{We use this piece to make the geometry of the rocker-bogie more structurally reinforced across its lateral bending moment. However, these channels will need to have their corners cut out because of clearance issues further on.}. In total, you will need 4 of these modified pieces. Cut those now.

\begin{figure}[H]
  \centering
  \begin{minipage}[b]{0.45\textwidth}
    \includegraphics[width=\textwidth]{"Pictures/Fabrication/Channel Cut".PNG}
  \end{minipage}
  \hfill
  \begin{minipage}[b]{0.45\textwidth}
    \includegraphics[width=\textwidth]{"Pictures/Fabrication/3inChannel1".png}
  \end{minipage}
  \caption{Channel Cutting Dimensions}
  \label{channel dimensions}
\end{figure}


\subsection{Cutting the Aluminum Rod}

Take the 4 inch aluminum rod \textbf{S18} and cut it down to 3 inches long, indicated by Figure \ref{rod cut} \footnote{These cuts keep the rod from sticking out too far from either end of the rocker-bogie pivot joint}. In total, you will need 2 of these 3-inch aluminum rods. Cut those now.

\begin{figure}[H]
	\centering
	\includegraphics[width=0.5\textwidth]{"Pictures/Fabrication/3inch cut".png}
	\caption{Aluminum Rod cutting dimensions}
	\label{rod cut}
\end{figure}

\section{Mechanical/Structural Assembly}

The Rocker-Bogie assembly is what attaches your wheels to the rover's body and allows it to climb over obstacles.

\begin{table}[H]
    \centering
    \arrayrulecolor{lightgray}
    \sffamily\footnotesize
    \captionsetup{font={sf,bf}}
    \caption{Parts/Tools Necessary}
    \begin{tabular}{|N|Q|Q|I|N|Q|Q|I|}
        \hline
        \thead{Item} & \thead{Ref} & \thead{Qty} & \thead{Image} & \thead{Item} & \thead{Ref} & \thead{Qty} & \thead{Image} \\
        \hline
        3" Channel (Modified) & S2A & 4 & \partimg{../../../images/parts_list/S2A.png} & \#6 Washer & W1 & 28 & \partimg{../../../images/parts_list/W1.png} \\ \hline
        3.75" Channel & S3 & 4 & \partimg{../../../images/parts_list/S3.jpg} & 0.5" Nylon Washer & W3 & 12 & \partimg{../../../images/parts_list/W3.png} \\ \hline
        9" Channel & S5 & 2 & \partimg{../../../images/parts_list/S5.jpg} & \#6x1/4" Standoff & T1 & 32 & \partimg{../../../images/parts_list/T1.png} \\ \hline
        Pattern F Bracket & S7 & 8 & \partimg{../../../images/parts_list/S7.jpg} & \#6-32x3/8" Button Head Screw & B2 & 52 & \partimg{../../../images/parts_list/B2.png} \\ \hline
        0.5” Pillow Block & S11 & 4 & \partimg{../../../images/parts_list/S11.jpg} & \#6-32x1/2" Button Head Screw & B3 & 32 & \partimg{../../../images/parts_list/B3.png} \\ \hline
        0.5” Face Tapped Clamping Hub & S13 & 2 & \partimg{../../../images/parts_list/S13.png} & \#6-32 Locking Hex Net & B11 & 60 & \partimg{../../../images/parts_list/B11.png} \\ \hline
        0.5" x 3" Aluminum Tube (Modified) & S18A & 2 & \partimg{../../../images/parts_list/S18A.png} & Allen Key Set & & & \partimg{../../../images/parts_list/D2.jpeg} \\ \hline
        0.5” Collar Clamp & S22 & 4 & \partimg{../../../images/parts_list/S22.png} & 5/16 Wrench & & & \partimg{../../../images/parts_list/D1.jpg} \\ \hline
    \end{tabular}
\end{table}


\begin{enumerate}
\item \textbf{Begin the Rocker Attachment:}  Attach 9 inch channel \textbf{S5} and the a modified 3 inch channel \textbf{S2A} to pattern F bracket \textbf{S7} using screws \textbf{B2} and hex nuts \textbf{B11} making sure to put a 6-32 washer \textbf{W1} in between the channels and the bracket. Make sure that the cut modification in \textbf{S2A} is angled towards the 9 inch channel as shown in Figure \ref {RB2}.

\begin{figure}[H]
  	\centering
  	\begin{minipage}[b]{0.30\textwidth}
    		\includegraphics[width=\textwidth]{"Pictures/Rocker-Bogie/RB Step 1".PNG}
  	\end{minipage}
  	\hfill
  	\begin{minipage}[b]{0.30\textwidth}
    		\includegraphics[width=\textwidth]{"Pictures/Rocker-Bogie/RB Step 1A".PNG}
  	\end{minipage}
    	\hfill
  	\begin{minipage}[b]{0.30\textwidth}
    		\includegraphics[width=\textwidth]{"Pictures/Rocker-Bogie/RB Step 2".PNG}
  	\end{minipage}
  	\caption{Begin the Rocker attachment}
\end{figure}


%\begin{figure}[H]
%  \centering
%  \begin{minipage}[b]{0.45\textwidth}
%    \includegraphics[width=\textwidth]{"Pictures/Rocker-Bogie/RB Step 1".PNG}
%  \end{minipage}
%  \hfill
%  \begin{minipage}[b]{0.45\textwidth}
%    \includegraphics[width=\textwidth]{"Pictures/Rocker-Bogie/RB Step 2".PNG}
%  \end{minipage}
%  \caption{Begin the Rocker attachment}
%\end{figure}

\item \textbf{Start the Bogie pivot joint:} Put rod \textbf{S18A} though the last hole in the pattern F bracket. Then,  assemble the pivot joint by attaching another modified 3 inch channel \textbf{S2A} and another pattern F bracket using washers \textbf{W3}, collar clamp \textbf{S22}, standoffs \textbf{T2}, and screws \textbf{B3}. Make sure to use the 2nd from the bottom hole on the 2nd pattern F bracket. See Figure \ref{RB2}:


\begin{figure}[H]
  \centering
  \begin{minipage}[b]{0.45\textwidth}
    \includegraphics[width=\textwidth]{"Pictures/Rocker-Bogie/RB Step 3".PNG}
  \end{minipage}
  \hfill
  \begin{minipage}[b]{0.45\textwidth}
    \includegraphics[width=\textwidth]{"Pictures/Rocker-Bogie/RB Step 4".PNG}
  \end{minipage}
  \caption{Start Rocker-Bogie pivot joint}
  \label{RB2}
\end{figure}




\item \textbf{Attachment for Front Corner:} Attach 3.75 inch channel \textbf{S3} to the top hole of the pattern F bracket using spacers \textbf{T1}, screws \textbf{B3}, and hex nuts \textbf{B11} as shown in Figure \ref{attaching channel corner}.

\begin{figure}[H]
  \centering
  \begin{minipage}[b]{0.45\textwidth}
    \includegraphics[width=\textwidth]{"Pictures/Rocker-Bogie/RB Step 5".PNG}
  \end{minipage}
  \hfill
  \begin{minipage}[b]{0.45\textwidth}
    \includegraphics[width=\textwidth]{"Pictures/Rocker-Bogie/Rb Step 6".png}
  \end{minipage}
  \caption{Attaching channel to corner steering}
  \label{attaching channel corner}
\end{figure}


\item \textbf{Attachment for middle Wheel:} Attach another 3.75 inch channel \textbf{S3} to the bottom hole of the pattern F bracket using spacers \textbf{T1}, screws \textbf{B3}, and hex nuts \textbf{B11} as shown in Figure \ref{attaching middle wheel}.

\begin{figure}[H]
  \centering
  \begin{minipage}[b]{0.45\textwidth}
    \includegraphics[width=\textwidth]{"Pictures/Rocker-Bogie/RB Step 8 Real".PNG}
  \end{minipage}
  \hfill
  \begin{minipage}[b]{0.45\textwidth}
    \includegraphics[width=\textwidth]{"Pictures/Rocker-Bogie/Rb Step 9".png}
  \end{minipage}
  \caption{Attachment for middle wheel}
  \label{attaching middle wheel}
\end{figure}


\item \textbf{Opposite edge of Rocker-Bogie Joint:} Attach another Pattern F bracket \textbf{S7} to the opposite side of the channels using washers \textbf{W1} and \textbf{W3}, screws \textbf{B3}, and hex nuts \textbf{B8} as shown in Figure \ref{start other edge rocker bogie}.

\begin{figure}[H]
	\centering
	\includegraphics[width=.65\textwidth]{"Pictures/Rocker-Bogie/RB Step 9_5".png}
	\caption{Start the other side of Rocker-Bogie joint}
	\label{start other edge rocker bogie}
\end{figure}

\item \textbf{Other edge of Rocker-Bogie Joint Cont: } Using washers \textbf{W1}, collar clamp \textbf{S22},  spacers \textbf{T1}, and screws \textbf{B10} attach pattern F bracket to the final connecting channels as shown in Figure \ref{other edge rocker bogie}.

\begin{figure}[H]
  \centering
  \begin{minipage}[b]{0.45\textwidth}
    \includegraphics[width=\textwidth]{"Pictures/Rocker-Bogie/RB Step 10 Real".PNG}
  \end{minipage}
  \hfill
  \begin{minipage}[b]{0.45\textwidth}
    \includegraphics[width=\textwidth]{"Pictures/Rocker-Bogie/RB Step 11".png}
  \end{minipage}
  \caption{Other edge of Rocker-Bogie joint}
  \label{other edge rocker bogie}
\end{figure}

\item \textbf{Checkpoint:} At this point your rocker bogie joint should look like Figure \ref{checkpoint}:

\begin{figure}[H]
  \centering
  \begin{minipage}[b]{0.45\textwidth}
    \includegraphics[width=\textwidth]{"Pictures/Rocker-Bogie/RB Final 1".PNG}
  \end{minipage}
  \hfill
  \begin{minipage}[b]{0.45\textwidth}
    \includegraphics[width=\textwidth]{"Pictures/Rocker-Bogie/RB Final 2".png}
  \end{minipage}
  \caption{Rocker-Bogie Checkpoint}
  \label{checkpoint}
\end{figure}

\item \textbf{Attach Bearing Blocks:} Attach two Pillow Bearing Blocks \textbf{S11} using screws \textbf{B2} and hex huts \textbf{B11} as shown in Figure \ref{pillow bearing block}. The bearing blocks should be aligned with the 5th large hole back from the front of the 9-inch U channel.

\begin{figure}[H]
  \centering
  \begin{minipage}[b]{0.45\textwidth}
    \includegraphics[width=\textwidth]{"Pictures/Rocker-Bogie/RB Step 13".PNG}
  \end{minipage}
  \hfill
  \begin{minipage}[b]{0.45\textwidth}
    \includegraphics[width=\textwidth]{"Pictures/Rocker-Bogie/RB Step 14".png}
  \end{minipage}
  \caption{Pillow bearing block attachment}
  \label{pillow bearing block}
\end{figure}

\item \textbf{Attach Clamping Hub:} Attach the Clamping Hub \textbf{S13} using screws \textbf{B2} and hex nuts \textbf{B11} as shown in Figure \ref{clamping hub}. The clamping hub should be centered above where you just placed the pillow bearing blocks.

\begin{figure}[H]
  \centering
  \begin{minipage}[b]{0.45\textwidth}
    \includegraphics[width=\textwidth]{"Pictures/Rocker-Bogie/RB Step 15".PNG}
  \end{minipage}
  \hfill
  \begin{minipage}[b]{0.45\textwidth}
    \includegraphics[width=\textwidth]{"Pictures/Rocker-Bogie/RB Step 16".png}
  \end{minipage}
  \caption{Clamping hub attachment}
  \label{clamping hub}
\end{figure}

Your rocker-bogie suspension joint is now complete!  Repeat all of section 2 in this document to create the suspension for the other side of the rover.


\end{enumerate}


\end{document}
