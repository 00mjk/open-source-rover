\documentclass[12pt]{article}
\usepackage[margin=1in]{geometry}
\usepackage{setspace}
\usepackage{graphicx}
\usepackage{subcaption}
\usepackage{amsmath}
\usepackage{color}
\usepackage{hyperref}
\usepackage{multicol}
\usepackage{framed}
\usepackage{xcolor}
\usepackage{wrapfig}
\usepackage{float}
\usepackage{fancyhdr}
\usepackage{verbatim}
\pagestyle{fancy}
\lfoot{\textbf{Open Source Rover Mechanical Assembly Manual}}
\rfoot{Page \thepage}
\lhead{\textbf{\leftmark}}
\rhead{\textbf{\rightmark}}
\cfoot{}
\renewcommand{\footrulewidth}{1.8pt}
\renewcommand{\headrulewidth}{1.8pt}
\doublespacing
\setlength{\parindent}{1cm}

\begin{document}


\title{Open Source Rover}
\author{Rocker-Bogie Assembly Instructions}

\makeatletter         
\def\@maketitle{
\begin{center}	
	\makebox[\textwidth][c]{ \includegraphics[width=0.8\paperwidth]{"Pictures/Rocker-Bogie/Rocker Bogie Title".png}}
	{\Huge \bfseries \sffamily \@title }\\[4ex] 
	{\huge \bfseries \sffamily \@author}\\[4ex] 
	\includegraphics[width=.65\linewidth]{"Pictures/Misc/JPL logo".png}
\end{center}}
\makeatother

\maketitle

% Introduction

\newpage

\tableofcontents

\newpage
\section{Machining/Fabrication Steps}

\begin{figure}[H]
	\centering
	\includegraphics[width=1\textwidth]{"Pictures/Parts-Cutting".PNG}
\end{figure}

\subsection{Aluminum Channels Cutting}
On piece \textbf{S2} measure the distance shown in \ref {channel dimensions} from the edge and mark a straight line from the corner to that point, and cut off the corner along that line \footnote{To make the geometry of the rocker-bogie and structurally more reinforced in it's bending moment we will add channels to the brackets, but they will need to have their corners cut out for clearance issues}

\begin{figure}[H]
  \centering
  \begin{minipage}[b]{0.45\textwidth}
    \includegraphics[width=\textwidth]{"Pictures/Fabrication/Channel Cut".PNG}
  \end{minipage}
  \hfill
  \begin{minipage}[b]{0.45\textwidth}
    \includegraphics[width=\textwidth]{"Pictures/Fabrication/3inChannel1".png}
  \end{minipage}
  \caption{Channel Cutting Dimensions}
  \label{channel dimensions}
\end{figure}


\subsection{Aluminum Rod Cutting}

Take the 4 inch aluminum rod \textbf{S18} and cut it down to 3 inches long, indicated by Figure \ref{rod cut} \footnote{This cut is purely to keep the rod from sticking out too far either end of the rocker-bogie pivot joint}.

\begin{figure}[H]
	\centering
	\includegraphics[width=0.5\textwidth]{"Pictures/Fabrication/3inch cut".png}
	\caption{Aluminum Rod cutting dimensions}
	\label{rod cut}
\end{figure}

\section{Mechanical/Structural Assembly}

The Rocker-Bogie assembly is what attaches your wheels, and enables the geometry to climb.

\begin{figure}[H]
	\centering
	\includegraphics[width=1\textwidth]{"Pictures/Rocker-Bogie/Rocker Bogie Parts".PNG}
\end{figure}




\begin{enumerate}
\item \textbf{Begin the Rocker Attachment:}  Attach 9 inch channel \textbf{S5} and the a modified 3 inch channel \textbf{S2A} to pattern F bracket \textbf{S7} making sure to put a 6-32 washer \textbf{W1} in between them, and use screws \textbf{B2} and hex nuts \textbf{B11}. Make sure that the cut modification in \textbf{S2A} is angled towards the 9 inch channel, this can been seen better in Figure \ref {RB2} clearer if necessary. 

\begin{figure}[H]
  	\centering
  	\begin{minipage}[b]{0.30\textwidth}
    		\includegraphics[width=\textwidth]{"Pictures/Rocker-Bogie/RB Step 1".PNG}
  	\end{minipage}
  	\hfill
  	\begin{minipage}[b]{0.30\textwidth}
    		\includegraphics[width=\textwidth]{"Pictures/Rocker-Bogie/RB Step 1A".PNG}
  	\end{minipage}
    	\hfill
  	\begin{minipage}[b]{0.30\textwidth}
    		\includegraphics[width=\textwidth]{"Pictures/Rocker-Bogie/RB Step 2".PNG}
  	\end{minipage}
  	\caption{Begin the Rocker attachment}
\end{figure}


%\begin{figure}[H]
%  \centering
%  \begin{minipage}[b]{0.45\textwidth}
%    \includegraphics[width=\textwidth]{"Pictures/Rocker-Bogie/RB Step 1".PNG}
%  \end{minipage}
%  \hfill
%  \begin{minipage}[b]{0.45\textwidth}
%    \includegraphics[width=\textwidth]{"Pictures/Rocker-Bogie/RB Step 2".PNG}
%  \end{minipage}
%  \caption{Begin the Rocker attachment}
%\end{figure}

\item \textbf{Start the Bogie pivot joint:} Put rod \textbf{S18A} though the last hole in the current pattern F bracket, and then using washers \textbf{W3}, collar clamp \textbf{S22}, standoffs \textbf{T2}, screws \textbf{B4}, another modified 3 inch channel \textbf{S2A}, and another pattern F bracket assemble the pivot joint. Make sure to use the 2nd from the bottom hole on the 2nd pattern F bracket. 


\begin{figure}[H]
  \centering
  \begin{minipage}[b]{0.45\textwidth}
    \includegraphics[width=\textwidth]{"Pictures/Rocker-Bogie/RB Step 3".PNG}
  \end{minipage}
  \hfill
  \begin{minipage}[b]{0.45\textwidth}
    \includegraphics[width=\textwidth]{"Pictures/Rocker-Bogie/RB Step 4".PNG}
  \end{minipage}
  \caption{Start Rocker-Bogie pivot joint}
  \label{RB2}
\end{figure}




\item \textbf{Attachment for Front Corner: } Attach 3.75 inch channel \textbf{S3} to the top hole of the pattern F bracket. Use spacers \textbf{T1}, screws \textbf{B4}, and hexs nuts \textbf{B11}. 

\begin{figure}[H]
  \centering
  \begin{minipage}[b]{0.45\textwidth}
    \includegraphics[width=\textwidth]{"Pictures/Rocker-Bogie/RB Step 5".PNG}
  \end{minipage}
  \hfill
  \begin{minipage}[b]{0.45\textwidth}
    \includegraphics[width=\textwidth]{"Pictures/Rocker-Bogie/Rb Step 6".png}
  \end{minipage}
  \caption{Attaching channel to corner steering }
\end{figure}


\item \textbf{Attachment for middle Wheel: } Attach 3.75 inch channel \textbf{S3} to the bottom hole of the pattern F bracket just attached. Use spacers \textbf{T1}, screws \textbf{B4}, and hex nuts \textbf{B11}. 

\begin{figure}[H]
  \centering
  \begin{minipage}[b]{0.45\textwidth}
    \includegraphics[width=\textwidth]{"Pictures/Rocker-Bogie/RB Step 8 Real".PNG}
  \end{minipage}
  \hfill
  \begin{minipage}[b]{0.45\textwidth}
    \includegraphics[width=\textwidth]{"Pictures/Rocker-Bogie/Rb Step 9".png}
  \end{minipage}
  \caption{Attachment for middle wheel}
\end{figure}
 

\item \textbf{Other edge of Rocker-Bogie Joint:} Attach Pattern F bracket \textbf{S7} to the other side of the channels using washers \textbf{W1} and \textbf{W3}, screws \textbf{B4}, and hex nuts \textbf{B8}. 

\begin{figure}[H]
	\centering
	\includegraphics[width=.65\textwidth]{"Pictures/Rocker-Bogie/RB Step 9_5".png}
	\caption{Start the other side of Rocker-Bogie joint}
\end{figure}

\item \textbf{Other edge of Rocker-Bogie Joint Cont: } Using washers \textbf{W1}, collar clamp \textbf{S22},  spacers \textbf{T1}, and screws \textbf{B10} attach pattern F bracket to the final connecting channels. 

\begin{figure}[H]
  \centering
  \begin{minipage}[b]{0.45\textwidth}
    \includegraphics[width=\textwidth]{"Pictures/Rocker-Bogie/RB Step 10 Real".PNG}
  \end{minipage}
  \hfill
  \begin{minipage}[b]{0.45\textwidth}
    \includegraphics[width=\textwidth]{"Pictures/Rocker-Bogie/RB Step 11".png}
  \end{minipage}
  \caption{Other edge of Rocker-Bogie joint}
\end{figure}

\item \textbf{Checkpoint:} At this point your rocker bogie joint should look like the following. 

\begin{figure}[H]
  \centering
  \begin{minipage}[b]{0.45\textwidth}
    \includegraphics[width=\textwidth]{"Pictures/Rocker-Bogie/RB Final 1".PNG}
  \end{minipage}
  \hfill
  \begin{minipage}[b]{0.45\textwidth}
    \includegraphics[width=\textwidth]{"Pictures/Rocker-Bogie/RB Final 2".png}
  \end{minipage}
  \caption{Checkpoint}
\end{figure}

\item \textbf{Bearing Block Attachment: } Using Pillow Bearing Block \textbf{S13} and screws \textbf{B2} attach the clamping hub, which should be centered directly above where the pillow blocks were just places. 

\begin{figure}[H]
  \centering
  \begin{minipage}[b]{0.45\textwidth}
    \includegraphics[width=\textwidth]{"Pictures/Rocker-Bogie/RB Step 13".PNG}
  \end{minipage}
  \hfill
  \begin{minipage}[b]{0.45\textwidth}
    \includegraphics[width=\textwidth]{"Pictures/Rocker-Bogie/RB Step 14".png}
  \end{minipage}
  \caption{Pillow bearing block attachment}
\end{figure}

\item \textbf{Clamp Hub Attachment: } Using Clamping Hub \textbf{S11}, screws \textbf{B2}, and hex nut \textbf{B11} attach the pillow bearing blocks. The bearings should be centered about the 5th hole from the front of the rocker-bogie as shown below. 

\begin{figure}[H]
  \centering
  \begin{minipage}[b]{0.45\textwidth}
    \includegraphics[width=\textwidth]{"Pictures/Rocker-Bogie/RB Step 15".PNG}
  \end{minipage}
  \hfill
  \begin{minipage}[b]{0.45\textwidth}
    \includegraphics[width=\textwidth]{"Pictures/Rocker-Bogie/RB Step 16".png}
  \end{minipage}
  \caption{Clamping hub attachment}
\end{figure}


\end{enumerate}


\end{document}