\documentclass[12pt]{article}
\usepackage[margin=1in]{geometry}
\usepackage{setspace}
\usepackage{graphicx}
\usepackage{subcaption}
\usepackage{amsmath}
\usepackage{color}
\usepackage{hyperref}
\usepackage{multicol}
\usepackage{framed}
\usepackage{xcolor}
\usepackage{wrapfig}
\usepackage{float}
\usepackage{fancyhdr}
\usepackage{verbatim}
\usepackage{textcomp}
\usepackage{colortbl}
\usepackage{array, booktabs, caption}
\usepackage{makecell}

\pagestyle{fancy}
\lfoot{\textbf{Open Source Rover Mechanical Assembly Manual}}
\rfoot{Page \thepage}
\lhead{\textbf{\leftmark}}
\rhead{\textbf{\rightmark}}
\cfoot{}
\renewcommand{\footrulewidth}{1.8pt}
\renewcommand{\headrulewidth}{1.8pt}
\doublespacing
\setlength{\parindent}{1cm}
% Parts list tables
\renewcommand\theadfont{\bfseries}
\newcolumntype{I}{ >{\centering\arraybackslash} m{2cm} }  % part image
\newcolumntype{N}{ >{\centering\arraybackslash} m{3cm} }  % part name
\newcolumntype{Q}{ >{\centering\arraybackslash} m{0.75cm} }  % ref & qty


\begin{document}

\newcommand\partimg{\includegraphics[width=2cm,height=1.25cm,keepaspectratio]}


\title{Open Source Rover: Corner Steering Assembly Instructions}
\author{Authors: Michael Cox, Eric Junkins, Olivia Lofaro}

\makeatletter
\def\@maketitle{
\begin{center}
	\makebox[\textwidth][c]{ \includegraphics[width=0.8\paperwidth]{"Pictures/Corner Title".png}}
	{\Huge \bfseries \sffamily \@title }\\[3ex]
	{\Large \sffamily \@author}\\[3ex]
	\includegraphics[width=.65\linewidth]{"Pictures/JPL logo".png}
\end{center}}
\makeatother

\maketitle

\noindent {\footnotesize Reference herein to any specific commercial product, process, or service by trade name, trademark, manufacturer, or otherwise, does not constitute or imply its endorsement by the United States Government or the Jet Propulsion Laboratory, California Institute of Technology. \textcopyright  2018 California Institute of Technology. Government sponsorship acknowledged.}

% Introduction

\newpage

\tableofcontents


\section{3D printing}
First, print the 3D printed encoder mounts. The STL files for this are located in the Corner Steering Assembly folder, under 3D Printed Parts. If you do not have a 3D printer there are many online 3D printing services available. One such service is:

\begin{itemize}
	\item \href{https://www.makexyz.com}{https://www.makexyz.com}
\end{itemize}

To order these parts from Makexyz upload the STL file, select FDM under process, and PLA for Material, and then your desired color. \textbf{You will need a total of 4 of these encoder mount pieces}.

\section{Machining/Fabrication}
\subsection{Shaft Coupler cuts}

\begin{table}[H]
    \centering
    \arrayrulecolor{lightgray}
    \sffamily\footnotesize
    \captionsetup{font={sf,bf}}
    \caption{Parts/Tools Necessary}
    \begin{tabular}{|N|Q|Q|I|N|Q|Q|I|}
        \hline
        \thead{Item} & \thead{Ref} & \thead{Qty} & \thead{Image} & \thead{Item} & \thead{Ref} & \thead{Qty} & \thead{Image} \\
        \hline
        0.25” to 4mm Clamping Shaft Coupler & S23 & 4 & \partimg{../../../images/parts_list/S23.jpg} & Metal Hacksaw or Bandsaw & & & \partimg{../../../images/parts_list/D4.png} \\ \hline
        Vice Clamp or C Clamps & & & \partimg{../../../images/parts_list/D5.png} & & & & \\ \hline
    \end{tabular}
\end{table}

We use shaft couplers to attach a motor shaft to another shaft; in this particular instance the coupler attaches the corner steering motor to a 0.25 inch shaft. This system must hold the max torque that the corner steering system can see to keep the shafts from slipping or free spinning. However, in our testing there was too much material in these couplers to allow them to deform and fully grab around the two shafts. We also make another horizontal cut in the couplers which de-couples the clamping done by each screw. This allows the coupler to grab on each shaft independently which reduces slippage.


\begin{figure}[H]
  	\centering
  	\begin{minipage}[b]{0.35\textwidth}
   		 \includegraphics[width=\textwidth]{"Pictures/Shaft Coupler Cut".PNG}
	\end{minipage}
 	 \hfill
 	 \begin{minipage}[b]{0.55\textwidth}
  		  \includegraphics[width=\textwidth]{"Pictures/Shaft Coupler Cut 2".png}
  	\end{minipage}
 	 \caption{Cutting the Shaft couplers}
	\label{Shaft coupler cut}
\end{figure}

We used a clamp or vice to hold the coupler while making these cuts.  This allowed us to better align the cutting blade with the channel in the shaft couplers \textbf{S23} and protected our hands. Use the drawings in Figure \ref{Shaft coupler cut} to determine how deep to go with each cut.


\section{Mechanical/Structural Assembly}
The Corner Steering assembly contains the steering motors which allow the rover to utilize Ackerman steering. One important aspect of this assembly is the use of the bearing blocks. These blocks help to take forces on the motor shaft against the gearbox and minimize lateral moments applied against the motor shaft. By using the bearing blocks, we help protect the motor and motor shaft from these forces that could otherwise damage the motor and its gearbox. The lever arm for the corner steering system is much farther away from the motor than at the drive motors, where we can get away with directly attaching the load path to the motor shaft.

\begin{table}[H]
    \centering
    \arrayrulecolor{lightgray}
    \sffamily\footnotesize
    \captionsetup{font={sf,bf}}
    \caption{Parts/Tools Necessary}
    \begin{tabular}{|N|Q|Q|I|N|Q|Q|I|}
        \hline
        \thead{Item} & \thead{Ref} & \thead{Qty} & \thead{Image} & \thead{Item} & \thead{Ref} & \thead{Qty} & \thead{Image} \\
        \hline
        3" Channel & S2 & 4 & \partimg{../../../images/parts_list/S2.jpg} & Motor (Corner Motor) & E6 & 4 & \partimg{../../../images/parts_list/E6.jpg} \\ \hline
        Motor Mount F & S9 & 4 & \partimg{../../../images/parts_list/S9.jpg} & Absolute Encoder & E7 & 4 & \partimg{../../../images/parts_list/E7.jpg} \\ \hline
        0.25" Pillow Block & S10 & 8 & \partimg{../../../images/parts_list/S10.jpg} & \#6-32x1/4" Spacer & T1 & 16 & \partimg{../../../images/parts_list/T1.png} \\ \hline
        0.25" D-Shaft & S15 & 4 & \partimg{../../../images/parts_list/S15.jpg} & \#6-32x3/4" Threaded Standoff & T3 & 16 & \partimg{../../../images/parts_list/T3.jpg} \\ \hline
        0.25” to 4mm Clamping Shaft Coupler (Modified) & S23A & 4 & \partimg{../../../images/parts_list/S23.jpg} & \#6-32x1/4" Button Head Screw & B1 & 24 & \partimg{../../../images/parts_list/B1.png} \\ \hline
        1/8” Bore Pinion Gear & S27 & 4 & \partimg{../../../images/parts_list/S27.jpg} & \#6-32x3/4" Button Head Screw & B5 & 16 & \partimg{../../../images/parts_list/B5.png} \\ \hline
        3D Printed Encoder Mount & S31 & 4 & \partimg{../../../images/parts_list/S31.png} & Allen Key Set & & & \partimg{../../../images/parts_list/D2.jpeg} \\ \hline
    \end{tabular}
\end{table}

\begin{enumerate}
\item \textbf{Motor Mount:} Begin by mounting the motor \textbf{E6} to the 3 inch channel \textbf{S2} using the motor mount F \textbf{S9} and screws \textbf{B1} as shown.

\begin{figure}[H]
  \centering
  \begin{minipage}[b]{0.45\textwidth}
    \includegraphics[width=\textwidth]{"Pictures/CornerSt Step 1".PNG}
  \end{minipage}
  \hfill
  \begin{minipage}[b]{0.45\textwidth}
    \includegraphics[width=\textwidth]{"Pictures/CornerSt Step 2".PNG}
  \end{minipage}
  \caption{Corner Steering Step 1}
\end{figure}

\item \textbf{Shaft Coupler/Standoffs Attachment:} Attach the motor shaft to the 0.25 inch D-shaft \textbf{S15} using the shaft coupler \textbf{S23}. Also take the 0.75inch long standoffs \textbf{T5} and attach them to the bottom of the channel centered around the motor using screws \textbf{B1} as shown in Figure \ref{corner steering 2}.

\begin{figure}[H]
  \centering
  \begin{minipage}[b]{0.45\textwidth}
    \includegraphics[width=\textwidth]{"Pictures/CornerSt Step 3".PNG}
  \end{minipage}
  \hfill
  \begin{minipage}[b]{0.45\textwidth}
    \includegraphics[width=\textwidth]{"Pictures/CornerSt Step 4".PNG}
  \end{minipage}
  \caption{Corner Steering Step 2}
  \label{corner steering 2}
\end{figure}

\item \textbf{Encoder Mount:} Press the 0.25inch Spacers \textbf{T1} into the 3D printed encoder mount \textbf{S31}. If they do not fit, you can drill or file out the holes slightly until the spacers fit as shown in Figure \ref{corner steering 3} (The size and tolerance of the holes will vary slightly based on different 3D printers and materials). Attach the encoder mount and bearings to the threaded standoffs \textbf{T3} using screws \textbf{B5} and 0.25 inch pillow bearing blocks \textbf{S10}.

\begin{figure}[H]
  \centering
  \begin{minipage}[b]{0.45\textwidth}
    \includegraphics[width=\textwidth]{"Pictures/CornerSt Step 5".PNG}
  \end{minipage}
  \hfill
  \begin{minipage}[b]{0.45\textwidth}
    \includegraphics[width=\textwidth]{"Pictures/CornerSt Step 6".PNG}
  \end{minipage}
  \caption{Corner Steering Step 3}
  \label{corner steering 3}
\end{figure}

\item \textbf{Encoder:} Attach the encoder \textbf{E7} to the encoder mount \textbf{S31} and then attach the 12-tooth gear \textbf{S27} to the encoder shaft. We will worry about its exact position later on. You should now have one finished corner steering assembly. Repeat the steps above to build the other 3 corner steering assemblies. \textbf{Note that unlike other assemblies, this assembly isn't identical for all four of them! You will need to build two versions where the encoder mount is mirrored about the motor shaft (see Figure \ref{corner steering 4} for an example).}

\begin{figure}[H]
  \centering
  \begin{minipage}[b]{0.45\textwidth}
    \includegraphics[width=\textwidth]{"Pictures/CornerSt Step 7".PNG}
  \end{minipage}
  \hfill
  \begin{minipage}[b]{0.45\textwidth}
    \includegraphics[width=\textwidth]{"Pictures/CornerSt Final".PNG}
  \end{minipage}
  \caption{Corner Steering Step 4}
  \label{corner steering 4}
\end{figure}

\end{enumerate}

\end{document}
