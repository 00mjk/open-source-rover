\documentclass[12pt]{article}
\usepackage[margin=1in]{geometry}
\usepackage{setspace}
\usepackage{graphicx}
\usepackage{subcaption}
\usepackage{amsmath}
\usepackage{color}
\usepackage{hyperref}
\usepackage{multicol}
\usepackage{framed}
\usepackage{xcolor}
\usepackage{wrapfig}
\usepackage{float}
\usepackage{fancyhdr}
\usepackage{verbatim}
\pagestyle{fancy}
\lfoot{\textbf{Open Source Rover Mechanical Assembly Manual}}
\rfoot{Page \thepage}
\lhead{\textbf{\leftmark}}
\rhead{\textbf{\rightmark}}
\cfoot{}
\renewcommand{\footrulewidth}{1.8pt}
\renewcommand{\headrulewidth}{1.8pt}
\doublespacing
\setlength{\parindent}{1cm}

\begin{document}


\title{Open Source Rover: Head Assembly Instructions}
\author{Author: Eric Junkins}

\makeatletter         
\def\@maketitle{
\begin{center}	
	\makebox[\textwidth][c]{ \includegraphics[width=0.6\paperwidth]{"Pictures/Head final".png}}
	{\Huge \bfseries \sffamily \@title }\\[3ex] 
	{\Large \sffamily \@author}\\[3ex] 
	\includegraphics[width=.65\linewidth]{"Pictures/JPL logo".png}
\end{center}}
\makeatother

\maketitle

\noindent {\footnotesize Reference herein to any specific commercial product, process, or service by trade name, trademark, manufacturer, or otherwise, does not constitute or imply its endorsement by the United States Government or the Jet Propulsion Laboratory, California Institute of Technology.}

\noindent {\footnotesize \textcopyright  2018 California Institute of Technology. Government sponsorship acknowledged}


% Introduction
\newpage


\tableofcontents

\newpage

\section{3D printing}
There are a few components that need to be 3D printed to make the head assembly. Provided are a the STL files necessary for these prints, as well as a few different orientations of printing pads to help with thermal contraction if necessary as shown in Figure \ref{pads}. These pads help hold the corners down and reduce the warping of the pieces. A file is also included with no pads for all files as well. 

\begin{figure}[H]
  \centering
  \begin{minipage}[b]{0.45\textwidth}
    \includegraphics[width=\textwidth]{"Pictures/Head w horz pads".PNG}
  \end{minipage}
  \hfill
  \begin{minipage}[b]{0.45\textwidth}
    \includegraphics[width=\textwidth]{"Pictures/Head w vert pads".png}
  \end{minipage}
  \caption{Printing Pads}
  \label{pads}
\end{figure}

If you do not have a 3D printer there are a number of online 3D printing services available, an example of which can be found at:

\begin{itemize}
	\item \href{https://www.makexyz.com/}{https://www.makexyz.com/}
\end{itemize}

\section{Mechanical Assembly}

\begin{figure}[H]
	\centering
	\includegraphics[width=1\textwidth]{"Pictures/Parts-Head".PNG}
\end{figure}

\begin{enumerate}

	\item \textbf{Insert the LED matrix into the head:} Start by taking the LED Matrix \textbf{E8} and inserting it on the ledges in the two sides of the 3D printed head. Using Screws \textbf{B8} screw together the two sides of the head \footnote{The size of the holes should be such that you can use an allen wrench to "thread" the 3D printed holes. If it is too tight you can use a drill/file to very slightly open up the hole if the screw will not thread fit}.


\begin{figure}[H]
	\centering
  	\begin{minipage}[b]{0.45\textwidth}
		\includegraphics[width=\textwidth]{"Pictures/Head step 1".PNG}
  	\end{minipage}
  	\hfill
  	\begin{minipage}[b]{0.45\textwidth}
    		\includegraphics[width=\textwidth]{"Pictures/Head step 2".png}
  	\end{minipage}
  	\caption{Inserting LED Matrix}
  	\label{LED}
\end{figure}

	\item \textbf{Attach the PVC  Clamping Hub:} Using screws \textbf{B2} attach the 1 Inch PVC Clamping Hub \textbf{S24} to the bottom of the head. 

\begin{figure}[H]
	\centering
	\includegraphics[width=0.45\textwidth]{"Pictures/Head step 3".PNG}
	\caption{Attach the PVC Clamping Hub}
\end{figure}

	\item \textbf{Mounting the Logic Shifter PCB:} Using screws \textbf{B8} and hexnuts \textbf{B12} attach the Logic Shifter PCB to the support structure in the 3D printed head. 

\begin{figure}[H]
	\centering
	\includegraphics[width=0.45\textwidth]{"Pictures/Head step 4".PNG}
	\caption{Mounting the Logic Shifter PCB}
\end{figure}

	\item \textbf{Connecting to back plane:} Using screws \textbf{B8} attach the two back pieces together. 

\begin{figure}[H]
	\centering
  	\begin{minipage}[b]{0.45\textwidth}
		\includegraphics[width=\textwidth]{"Pictures/Head step 5".PNG}
  	\end{minipage}
  	\hfill
  	\begin{minipage}[b]{0.45\textwidth}
    		\includegraphics[width=\textwidth]{"Pictures/Head step 6".png}
  	\end{minipage}
  	\caption{Back Plane connecting}
  	\label{back}
\end{figure}

	\item \textbf{Attaching the Back plane:} Using screws \textbf{B8} now attach the back plane to the rest of the head structure. 

\begin{figure}[H]
	\centering
	\includegraphics[width=0.45\textwidth]{"Pictures/Head step 7".PNG}
	\caption{Attaching the Back plane}
\end{figure}

	\item \textbf{Attach the PVC pipe:} Slide the length of PVC pipe you want for the neck length into the PVC clamping hub and tighten the screws the secure it.

\begin{figure}[H]
	\centering
  	\begin{minipage}[b]{0.45\textwidth}
		\includegraphics[width=\textwidth]{"Pictures/Head step 8".PNG}
  	\end{minipage}
  	\hfill
  	\begin{minipage}[b]{0.45\textwidth}
    		\includegraphics[width=\textwidth]{"Pictures/Head final".png}
  	\end{minipage}
  	\caption{Attaching the PVC pipe}
  	\label{pvc}
\end{figure}

\end{enumerate}

\end{document}