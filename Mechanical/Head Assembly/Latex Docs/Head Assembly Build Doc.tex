\documentclass[12pt]{article}
\usepackage[margin=1in]{geometry}
\usepackage{setspace}
\usepackage{graphicx}
\usepackage{subcaption}
\usepackage{amsmath}
\usepackage{color}
\usepackage{hyperref}
\usepackage{multicol}
\usepackage{framed}
\usepackage{xcolor}
\usepackage{wrapfig}
\usepackage{float}
\usepackage{fancyhdr}
\usepackage{verbatim}
\usepackage{colortbl}
\usepackage{array, booktabs, caption}
\usepackage{makecell}

\pagestyle{fancy}
\lfoot{\textbf{Open Source Rover Mechanical Assembly Manual}}
\rfoot{Page \thepage}
\lhead{\textbf{\leftmark}}
\rhead{\textbf{\rightmark}}
\cfoot{}
\renewcommand{\footrulewidth}{1.8pt}
\renewcommand{\headrulewidth}{1.8pt}
\doublespacing
\setlength{\parindent}{1cm}

% Parts list tables
\renewcommand\theadfont{\bfseries}
\newcolumntype{I}{ >{\centering\arraybackslash} m{2cm} }  % part image
\newcolumntype{N}{ >{\centering\arraybackslash} m{3cm} }  % part name
\newcolumntype{Q}{ >{\centering\arraybackslash} m{0.75cm} }  % ref & qty


\begin{document}

\newcommand\partimg{\includegraphics[width=2cm,height=1.25cm,keepaspectratio]}


\title{Open Source Rover: Head Assembly Instructions}
\author{Authors: Michael Cox, Eric Junkins, Olivia Lofaro}

\makeatletter
\def\@maketitle{
\begin{center}
	\makebox[\textwidth][c]{ \includegraphics[width=0.6\paperwidth]{"Pictures/Head final".png}}
	{\Huge \bfseries \sffamily \@title }\\[3ex]
	{\Large \sffamily \@author}\\[3ex]
	\includegraphics[width=.65\linewidth]{"Pictures/JPL logo".png}
\end{center}}
\makeatother

\maketitle

\noindent {\footnotesize Reference herein to any specific commercial product, process, or service by trade name, trademark, manufacturer, or otherwise, does not constitute or imply its endorsement by the United States Government or the Jet Propulsion Laboratory, California Institute of Technology. \textcopyright  2018 California Institute of Technology. Government sponsorship acknowledged.}


% Introduction
\newpage


\tableofcontents

\newpage

\section{3D printing}
There are a few components that need to be 3D printed to make the head assembly. You can find the STL files necessary for these prints in the Mechanical/Head Assembly/3D Printed Parts folder of the repository.  Based on difficulties we've seen with printing these pieces, we've also included a few different orientations of "printing pads" to help with thermal contraction as shown in Figure \ref{pads}. These pads help hold the corners down and reduce the warping of the pieces.

\begin{figure}[H]
  \centering
  \begin{minipage}[b]{0.45\textwidth}
    \includegraphics[width=\textwidth]{"Pictures/Head w horz pads".PNG}
  \end{minipage}
  \hfill
  \begin{minipage}[b]{0.45\textwidth}
    \includegraphics[width=\textwidth]{"Pictures/Head w vert pads".png}
  \end{minipage}
  \caption{Printing Pads}
  \label{pads}
\end{figure}

If you do not have a 3D printer there are a number of online 3D printing services available, an example of which can be found at:

\begin{itemize}
	\item \href{https://www.makexyz.com/}{https://www.makexyz.com/}
\end{itemize}

(If you decide to use a 3D printing service, we recommend sending them the files \textit{without} the printing pads.)

\section{Machining/Fabrication}
\subsection{Cutting the PVC Pipe}


\begin{table}[H]
    \centering
    \arrayrulecolor{lightgray}
    \sffamily\footnotesize
    \captionsetup{font={sf,bf}}
    \caption{Parts/Tools Necessary}
    \begin{tabular}{|N|Q|Q|I|N|Q|Q|I|}
        \hline
        \thead{Item} & \thead{Ref} & \thead{Qty} & \thead{Image} & \thead{Item} & \thead{Ref} & \thead{Qty} & \thead{Image} \\
        \hline
        2" PCV Pipe & S29 & 1 & \partimg{../../../images/parts_list/S29.png} & Vice or V-Clamps & D8 & & \partimg{../../../images/parts_list/D8.png} \\ \hline
        HackSaw or Bandsaw & D4 & & \partimg{../../../images/parts_list/D4.png} & & & & \\ \hline
    \end{tabular}
\end{table}


Take the PVC pipe \textbf{S29} (this will be the "neck" of the rover) and cut it to your desired length. For reference, we cut our "neck" PVC pipe to be roughly 6 inches long.


\section{Mechanical Assembly}

\begin{table}[H]
    \centering
    \arrayrulecolor{lightgray}
    \sffamily\footnotesize
    \captionsetup{font={sf,bf}}
    \caption{Parts/Tools Necessary}
    \begin{tabular}{|N|Q|Q|I|N|Q|Q|I|}
        \hline
        \thead{Item} & \thead{Ref} & \thead{Qty} & \thead{Image} & \thead{Item} & \thead{Ref} & \thead{Qty} & \thead{Image} \\
        \hline
        Head Left Back & S33A & 1 & \partimg{../../../images/parts_list/S33A.png} & \#6-32x3/8" Button Head Screw & B2 & 4 & \partimg{../../../images/parts_list/B2.png} \\ \hline
        Head Right Back & S33C & 1 & \partimg{../../../images/parts_list/S33C.png} & \#4-40x1/4" Button Head Screw & B8 & 12 & \partimg{../../../images/parts_list/B8.png} \\ \hline
        Head Left Side & S33B & 1 & \partimg{../../../images/parts_list/S33B.png} & \#4-40 Locking Hex Nut & B12 & 4 & \partimg{../../../images/parts_list/B12.png} \\ \hline
        Head Right Side & S33D & 1 & \partimg{../../../images/parts_list/S33D.png} & LED Matrix & E8 & 1 & \partimg{../../../images/parts_list/E8.jpg} \\ \hline
        Bore Clamping Hub for 1" PVC & S24 & 1 & \partimg{../../../images/parts_list/S24.jpg} & Allen Key Set & D2 & & \partimg{../../../images/parts_list/D2.jpeg} \\ \hline
        PVC Pipe (Modified) & S29A & 1 & \partimg{../../../images/parts_list/S29.png} & 5/16" Wrench & D1 & & \partimg{../../../images/parts_list/D1.jpg} \\ \hline
        Logic Shifter PCB & E11 & 1 & \partimg{../../../images/parts_list/ls2.PNG} & & & & \\ \hline
    \end{tabular}
\end{table}

\begin{enumerate}

	\item \textbf{Insert the LED matrix into the head:} Start by taking the LED Matrix \textbf{E8} and inserting it on the ledges in the left and right sides of the 3D printed head. Screw together the two sides of the head using screws \textbf{B8}.\footnote{The size of the holes should be such that you can use a hex key to "thread" the 3D printed holes. If it is too tight, you can use a drill/file to very slightly open up the hole}.


\begin{figure}[H]
	\centering
  	\begin{minipage}[b]{0.45\textwidth}
		\includegraphics[width=\textwidth]{"Pictures/Head step 1".PNG}
  	\end{minipage}
  	\hfill
  	\begin{minipage}[b]{0.45\textwidth}
    		\includegraphics[width=\textwidth]{"Pictures/Head step 2".png}
  	\end{minipage}
  	\caption{Inserting LED Matrix}
  	\label{LED}
\end{figure}

	\item \textbf{Attach the PVC  Clamping Hub:} Attach the 1 Inch PVC Clamping Hub \textbf{S24} to the bottom of the head using screws \textbf{B2}.

\begin{figure}[H]
	\centering
	\includegraphics[width=0.45\textwidth]{"Pictures/Head step 3".PNG}
	\caption{Attach the PVC Clamping Hub}
\end{figure}

	\item \textbf{Mounting the Logic Shifter PCB:} Attach the Logic Shifter PCB to the support structure in the 3D printed head using screws \textbf{B8} and hex nuts \textbf{B12}.

\begin{figure}[H]
	\centering
	\includegraphics[width=0.45\textwidth]{"Pictures/Head step 4".PNG}
	\caption{Mounting the Logic Shifter PCB}
\end{figure}

	\item \textbf{Connecting to back plane:} Attach the two back pieces together using screws \textbf{B8}.

\begin{figure}[H]
	\centering
  	\begin{minipage}[b]{0.45\textwidth}
		\includegraphics[width=\textwidth]{"Pictures/Head step 5".PNG}
  	\end{minipage}
  	\hfill
  	\begin{minipage}[b]{0.45\textwidth}
    		\includegraphics[width=\textwidth]{"Pictures/Head step 6".png}
  	\end{minipage}
  	\caption{Back Plane connecting}
  	\label{back}
\end{figure}

	\item \textbf{Attaching the Back plane:} Attach the back plane to the rest of the head structure using screws \textbf{B8}.

\begin{figure}[H]
	\centering
	\includegraphics[width=0.45\textwidth]{"Pictures/Head step 7".PNG}
	\caption{Attaching the Back plane}
\end{figure}

	\item \textbf{Attach the PVC pipe:} Slide the length of PVC pipe you want for the neck into the PVC clamping hub and tighten the screws the secure it.

\begin{figure}[H]
	\centering
  	\begin{minipage}[b]{0.45\textwidth}
		\includegraphics[width=\textwidth]{"Pictures/Head step 8".PNG}
  	\end{minipage}
  	\hfill
  	\begin{minipage}[b]{0.45\textwidth}
    		\includegraphics[width=\textwidth]{"Pictures/Head final".png}
  	\end{minipage}
  	\caption{Attaching the PVC pipe}
  	\label{pvc}
\end{figure}

\end{enumerate}

\end{document}
