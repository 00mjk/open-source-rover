\documentclass[12pt]{article}
\usepackage[margin=1in]{geometry}
\usepackage{setspace}
\usepackage{graphicx}
\usepackage{subcaption}
\usepackage{amsmath}
\usepackage{color}
\usepackage{hyperref}
\usepackage{multicol}
\usepackage{framed}
\usepackage{xcolor}
\usepackage{wrapfig}
\usepackage{float}
\usepackage{fancyhdr}
\usepackage{verbatim}
\usepackage{textcomp}
\pagestyle{fancy}
\lfoot{\textbf{Open Source Rover Mechanical Assembly Manual}}
\rfoot{Page \thepage}
\lhead{\textbf{\leftmark}}
\rhead{\textbf{\rightmark}}
\cfoot{}
\renewcommand{\footrulewidth}{1.8pt}
\renewcommand{\headrulewidth}{1.8pt}
\doublespacing
\setlength{\parindent}{1cm}

\begin{document}


\title{Open Source Rover: Body Assembly Instructions}
\author{Author: Eric Junkins}

\makeatletter         
\def\@maketitle{
\begin{center}	
	\makebox[\textwidth][c]{ \includegraphics[width=0.7\paperwidth]{"Pictures/Body/Body title".png}}
	{\Huge \bfseries \sffamily \@title }\\[3ex] 
	{\Large \sffamily \@author}\\[3ex] 
	\includegraphics[width=.65\linewidth]{"Pictures/Misc/JPL logo".png}
\end{center}}
\makeatother

\maketitle

\noindent {\footnotesize Reference herein to any specific commercial product, process, or service by trade name, trademark, manufacturer, or otherwise, does not constitute or imply its endorsement by the United States Government or the Jet Propulsion Laboratory, California Institute of Technology.}

\noindent {\footnotesize \textcopyright  2018 California Institute of Technology. Government sponsorship acknowledged}


% Introduction
\newpage


\tableofcontents

\newpage

\section{Machining/Fabrication Steps}
\subsection{Front/Back Aluminum Plates cuts}

First you need to cut the 4.5x12 inch aluminum plate. Follow Figure \ref{fb panel cut} to cut it correctly to dimensions listed. 

%\begin{figure}[H]
%  \centering
%  \begin{minipage}[b]{0.45\textwidth}
%    \includegraphics[width=\textwidth]{"Pictures/Fabrication/Front panel cut".PNG}
%  \end{minipage}
%  \hfill
%  \begin{minipage}[b]{0.45\textwidth}
%    \includegraphics[width=\textwidth]{"Pictures/Fabrication/Back panel cut".png}
%  \end{minipage}
%  \caption{Cutting the front and back panels of the body}
%  \label{fb panel cut}
%\end{figure}


\begin{figure}[H]
	\centering
	\includegraphics[width=0.7\textwidth]{"Pictures/Fabrication/Front panel cut".PNG}
  	\caption{Cutting the front and back panels of the body}
  	\label{fb panel cut}
\end{figure}


\subsection{Laser Cut parts}

In order to put the electronics inside the robot body we need an electronics board, one easy solution for this is to order a laser cut piece of acrylic. In the Body Assembly there is a folder for Laser cut parts. This contains two .DxF files which are 2D path files for the laser cutter. There are many inexpensive laser cutting website services, one example is:

\begin{itemize}
	\item \href{https://www.sculpteo.com}{https://www.sculpteo.com}
\end{itemize}

To get these parts from Sculpteo go to Laser cutting, and then upload these files, with mm selected as units. then hit next. Make sure scale is set to 100\%, change the material to Acrylic, have thickness to 1/8, and then select whatever color you wish. 


\subsection{9x12 Aluminum Plate Drilling}
Next we need to drill a hole in one of the 9x12 Aluminum plates \textbf{S35}. The reason for this is that we will need a hole of slightly greater than 0.5 in diameter for the differential pivot mount. There is already a hole drilled in the location we are wanting to use, it just needs to be widened. Start with the drill \# 23, drill the hole shown by the figure \ref{Drilling the Al plate}. Repeat this with drill sizes stepping up until you get to drill of 0.5 in or slightly greater. Take the 0.5 in hallow rod \textbf{S19} and make sure it spins freely in the hole created. If not drill slightly larger or sand/file until it spins with no resistance \footnote{The 0.5 in hallow rod must free spin while mounted inside the bearing blocks (See step 2.2 Differential pivot for example), it might help to follow 2.2 to test if there is enough clearance} 

\begin{figure}[H]
  \centering
  \begin{minipage}[b]{0.45\textwidth}
    \includegraphics[width=\textwidth]{"Pictures/Fabrication/9x12 Plate cut".PNG}
  \end{minipage}
  \hfill
  \begin{minipage}[b]{0.45\textwidth}
    \includegraphics[width=\textwidth]{"Pictures/Fabrication/9x12 Plate cut2".png}
  \end{minipage}
  \caption{Drilling the Aluminum Plate}
  \label{Drilling the Al plate}
\end{figure}

\newpage

\section{Mechanical/Structural Assembly}
\subsection{Chassis}

\begin{figure}[H]
	\centering
	\includegraphics[width=1\textwidth]{"Pictures/Body/Chassis Parts".PNG}
\end{figure}

\begin{enumerate}
\item \textbf{Attach the channels to the Top panel: } Take the modified 9x12 Aluminum plate \textbf{S35A} and attach the four 1.5inch channel connectors \textbf{S1} using screws \textbf{B2} at each of the corners as shown in \ref{channel to al plate}. Make sure to use the inner circle as your holes for attaching and not the outter ones, where there won't be enough clearance for the hex nut. 

\begin{figure}[H]
  	\centering
  	\begin{minipage}[b]{0.20\textwidth}
    		\includegraphics[width=\textwidth]{"Pictures/Body/Step 1".PNG}
  	\end{minipage}
  	\hfill
  	\begin{minipage}[b]{0.30\textwidth}
    		\includegraphics[width=\textwidth]{"Pictures/Body/Step 1 b".PNG}
  	\end{minipage}
    	\hfill
  	\begin{minipage}[b]{0.40\textwidth}
    		\includegraphics[width=\textwidth]{"Pictures/Body/Step 2".PNG}
  	\end{minipage}
  	\caption{Attaching channels to aluminum plate}
	\label{channel to al plate}
\end{figure}

\item \textbf{Attach the side panels: } Attach the 4.5x12 plates \textbf{S37} to the channels using screws \textbf{B2} and hex nuts \textbf{B11}, again using the middle circle of holes for the screws and hex nuts. 

\begin{figure}[H]
 	\centering
 	\begin{minipage}[b]{0.45\textwidth}
    		\includegraphics[width=\textwidth]{"Pictures/Body/Step 3 a".PNG}
  	\end{minipage}
  	\hfill
  	\begin{minipage}[b]{0.45\textwidth}
    		\includegraphics[width=\textwidth]{"Pictures/Body/Step 3 b".PNG}
  	\end{minipage}
  	\caption{Attach the side panels}
	\label{Body side panels}
\end{figure}

\item \textbf{Attach the PVC clamping hub:} Attach the 1' PVC bore clamping hub \textbf{S24} to the top plate of the body using screws \textbf{B1}. Figure \ref{pvc to top plate} shows where we chose to mount ours. 

\begin{figure}[H]
  \centering
  \begin{minipage}[b]{0.40\textwidth}
    \includegraphics[width=\textwidth]{"Pictures/Body/Step 11a".PNG}
  \end{minipage}
  \hfill
  \begin{minipage}[b]{0.50\textwidth}
    \includegraphics[width=\textwidth]{"Pictures/Body/Step 11b".PNG}
  \end{minipage}
  \caption{Attach the PVC clamp to top plate}
	\label{pvc to top plate}
\end{figure}

\end{enumerate}


\subsection{Differential Pivot Block}
The differential pivot is used to transfer weight off of the wheel that is currently climbing to the other front wheel, allowing the rover to climb easier. Additionally it serves as a second contact point such that the body does not rotate freely about the cross rod.

\begin{figure}[H]
	\centering
	\includegraphics[width=1\textwidth]{"Pictures/Body/Differential Parts".PNG}
\end{figure}

\begin{enumerate}
\item \textbf{Mount the pillow bearing blocks:} Using spacers \textbf{T1}, screws \textbf{B6} and hex nut \textbf{B11} mount pillow blocks \textbf{S11} to the top of the body in the indicated location, over the hole we drilled in the plate earlier.

\begin{figure}[H]
  \centering
  \begin{minipage}[b]{0.30\textwidth}
    \includegraphics[width=\textwidth]{"Pictures/Body/Step 5 a".PNG}
  \end{minipage}
  \hfill
  \begin{minipage}[b]{0.55\textwidth}
    \includegraphics[width=\textwidth]{"Pictures/Body/Step 5 b".PNG}
  \end{minipage}
  \caption{Mounting the pillow blocks}
\end{figure}


\end{enumerate}

\subsection{Electronics Board}

Next is preparing the electronics board. This is where all the electrical components live, it will have the Raspberry Pi, all 5 RoboClaw Motor controllers, and the voltage regulators. 

\begin{figure}[H]
	\centering
	\includegraphics[width=1\textwidth]{"Pictures/Body/Electronics Board Parts".PNG}
\end{figure}



\begin{enumerate}
\item \textbf{Attaching the Standoffs} There are a few different standoffs here. The height differences are to allow access to the micro USB port on the RoboClaws, and the Raspberry Pi has it's own different standoffs as it's metric. In the below picture the colors correspond as follows: \textcolor{green}{Green}:\textbf{T4}, \textcolor{blue}{Blue}:\textbf{T5}, \textcolor{pink}{Pink}:\textbf{T6}, \textcolor{cyan}{Cyan}:\textbf{T7}, \textcolor{yellow}{Yellow}:\textbf{T8}. Use the screw that corresponds to the spacer used.

\begin{figure}[H]
  \centering
  \begin{minipage}[b]{0.50\textwidth}
    \includegraphics[width=\textwidth]{"Pictures/Body/Step 9a".PNG}
  \end{minipage}
  \hfill
  \begin{minipage}[b]{0.35\textwidth}
    \includegraphics[width=\textwidth]{"Pictures/Body/Electronics Screw".PNG}
  \end{minipage}

  \caption{Electronics Board Step 1}
\end{figure}

\item \textbf{Mounting the Electronics:} Take the Raspberry Pi \textbf{E1}, RoboClaws \textbf{E2}, and voltage regulators \textbf{E3 and E4} and mount them in the locations designated by the below picture, again using the screws \textbf{B8 and B10} corresponding with each standoff. 

\begin{figure}[H]
\centering	
  \includegraphics[width=.65\linewidth]{"Pictures/Body/Step 9c".PNG}
  \caption{Electronics Board Step 2}
\end{figure}

\item \textbf{Mounting Electronics into Chassis: } Now that the electronics are on the board we can mount it into the Chassis. Using screws \textbf{B3}, 3x washers \textbf{W1} per corner, and hex nuts \textbf{B11} attach the electronics board to the chassis \footnote{The washers are to give small amount of space more needed to put the Voltage monitor in the system. \textcolor{red}{add pics of clearance}}

\begin{figure}[H]
  \centering
  \begin{minipage}[b]{0.40\textwidth}
    \includegraphics[width=\textwidth]{"Pictures/Body/Step 8b".PNG}
  \end{minipage}
  \hfill
  \begin{minipage}[b]{0.40\textwidth}
    \includegraphics[width=\textwidth]{"Pictures/Body/Step 8a".PNG}
  \end{minipage}
  \caption{Electronics Board Step 3}
\end{figure}
\end{enumerate}

\subsection{Closing the Body}

\begin{enumerate}
\begin{figure}[H]
	\centering
	\includegraphics[width=1\textwidth]{"Pictures/Body/Closing Body".PNG}
\end{figure}

\item \textbf{Attach the Dual Side Mounts: } Put Dual Side Mounts A \textbf{S17}, and screws \textbf{B1} in the locations shown in Figures \ref{Dual Side Mounts}. 

\begin{figure}[H]
  \centering
  \begin{minipage}[b]{0.40\textwidth}
    \includegraphics[width=\textwidth]{"Pictures/Body/Dual Side mounts".PNG}
  \end{minipage}
  \hfill
  \begin{minipage}[b]{0.40\textwidth}
    \includegraphics[width=\textwidth]{"Pictures/Body/Dual side mounts 2".PNG}
  \end{minipage}
  \caption{Dual Side Mount A locations}
  \label{Dual Side Mounts}
\end{figure}

\item \textbf{Attach the front/back panel: } Attach the Acrylic back panel \textbf{S37B} to the "back" of the body using screws \textbf{B2}, the side with the raspberry pi. The cutout should line up with the USB ports. Repeat this with the front of the body with \textbf{S37A}

\begin{figure}[H]
  \centering
  \begin{minipage}[b]{0.40\textwidth}
    \includegraphics[width=\textwidth]{"Pictures/Body/Back panel 1".PNG}
  \end{minipage}
  \hfill
  \begin{minipage}[b]{0.40\textwidth}
    \includegraphics[width=\textwidth]{"Pictures/Body/Back panel 2".PNG}
  \end{minipage}
  \caption{Mounting the front/back panels}
  \label{front/back panels}
\end{figure}

\item \textbf{Attach the bottom panel: } Attach the 9x12 Aluminum Plate \textbf{S35} using screws \textbf{B1}. 

At this point the body should be complete with the differential pivot mount, electronics, and chassis, and should look like the following. 

\begin{figure}[H]
  \centering
  \begin{minipage}[b]{0.45\textwidth}
    \includegraphics[width=\textwidth]{"Pictures/Body/Finished Body 1".PNG}
  \end{minipage}
  \hfill
  \begin{minipage}[b]{0.45\textwidth}
    \includegraphics[width=\textwidth]{"Pictures/Body/Finished Body 2".PNG}
  \end{minipage}
  \caption{Finished Body Assembly}
\end{figure}

\end{enumerate}

\end{document}