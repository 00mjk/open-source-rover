\documentclass[12pt]{article}
\usepackage[margin=1in]{geometry}
\usepackage{setspace}
\usepackage{graphicx}
\usepackage{subcaption}
\usepackage{amsmath}
\usepackage{color}
\usepackage{hyperref}
\usepackage{multicol}
\usepackage{framed}
\usepackage{xcolor}
\usepackage{wrapfig}
\usepackage{float}
\usepackage{fancyhdr}
\usepackage{verbatim}
\usepackage{colortbl}
\usepackage{array, booktabs, caption}
\usepackage{makecell}

\pagestyle{fancy}
\lfoot{\textbf{Open Source Rover Mechanical Assembly Manual}}
\rfoot{Page \thepage}
\lhead{\textbf{\leftmark}}
\rhead{\textbf{\rightmark}}
\cfoot{}
\renewcommand{\footrulewidth}{1.8pt}
\renewcommand{\headrulewidth}{1.8pt}
\doublespacing
\setlength{\parindent}{1cm}

% Parts list tables
\renewcommand\theadfont{\bfseries}
\newcolumntype{I}{ >{\centering\arraybackslash} m{2cm} }  % part image
\newcolumntype{N}{ >{\arraybackslash} m{3cm} }  % part name
\newcolumntype{Q}{ >{\centering\arraybackslash} m{0.75cm} }  % ref & qty


\begin{document}

\newcommand\partimg{\includegraphics[width=2cm,height=1.25cm,keepaspectratio]}


\title{Open Source Rover: Differential Pivot Assembly Instructions}
\author{Authors: Michael Cox, Eric Junkins, Olivia Lofaro}

\makeatletter
\def\@maketitle{
\begin{center}
	\makebox[\textwidth][c]{ \includegraphics[width=0.8\paperwidth]{"Pictures/Differential Pivot".png}}
	{\Huge \bfseries \sffamily \@title }\\[3ex]
	{\Large \sffamily \@author}\\[3ex]
	\includegraphics[width=.65\linewidth]{"Pictures/JPL logo".png}
\end{center}}
\makeatother

\maketitle

\noindent {\footnotesize Reference herein to any specific commercial product, process, or service by trade name, trademark, manufacturer, or otherwise, does not constitute or imply its endorsement by the United States Government or the Jet Propulsion Laboratory, California Institute of Technology. \textcopyright  2018 California Institute of Technology. Government sponsorship acknowledged.}


% Introduction
\newpage


\tableofcontents

\newpage


\section{Machining/Fabrication}
\subsection{Aluminum Rods: Cutting}

\begin{table}[H]
    \centering
    \arrayrulecolor{lightgray}
    \sffamily\footnotesize
    \captionsetup{font={sf,bf}}
    \caption{Parts/Tools Necessary}
    \begin{tabular}{|N|Q|Q|I|N|Q|Q|I|}
        \hline
        \thead{Item} & \thead{Ref} & \thead{Qty} & \thead{Image} & \thead{Item} & \thead{Ref} & \thead{Qty} & \thead{Image} \\
        \hline
        0.5"x3' Tube & S16 & 1 & \partimg{../../../images/parts_list/S16.png} & Metal Hacksaw or Bandsaw & & & \\ \hline
        0.5"x4" Tube & S18 & 4 & \partimg{../../../images/parts_list/S18.jpg} & Vice clamp or C Clamps & & & \\ \hline
    \end{tabular}
\end{table}

Take the 3 foot piece of 0.5 inch aluminum rod \textbf{S16} and cut it into one 15 inch piece and one 13.5 inch piece. These will now be referred to as the parts \textbf{S16A} and \textbf{S16B} respectively. In addition, take two of the 4 inch aluminum rods \textbf{S18} and cut them down to 3 inches in length as well.

\begin{figure}[H]
  \centering
  \begin{minipage}[b]{0.45\textwidth}
    \includegraphics[width=\textwidth]{"Pictures/15inch cut".PNG}
  \end{minipage}
  \hfill
  \begin{minipage}[b]{0.45\textwidth}
    \includegraphics[width=\textwidth]{"Pictures/3inch cut".png}
  \end{minipage}
  \caption{Aluminum Rod cutting}
  \label{Al dimensions}
\end{figure}

\subsection{Aluminum Rods: Drilling}

\begin{table}[H]
    \centering
    \arrayrulecolor{lightgray}
    \sffamily\footnotesize
    \captionsetup{font={sf,bf}}
    \caption{Parts/Tools Necessary}
    \begin{tabular}{|N|Q|Q|I|N|Q|Q|I|}
        \hline
        \thead{Item} & \thead{Ref} & \thead{Qty} & \thead{Image} & \thead{Item} & \thead{Ref} & \thead{Qty} & \thead{Image} \\
        \hline
        0.5" x 14" Aluminum Tube & S16 & 1 & \partimg{../../../images/parts_list/S16.png} & Hand Drill or Drill Press & & & \\ \hline
        0.5" x 3" Aluminum Tube & S18 & 2 & \partimg{../../../images/parts_list/S18.jpg} & Center punch or Starter drill bit & & & \\ \hline
        5 Hole Aluminum Beam & S21 & 8 & \partimg{../../../images/parts_list/S21.jpg} & Drill bit \#23 & & & \\ \hline
        Vice or V-Clamp & & & & & & &  \\ \hline
    \end{tabular}
\end{table}

The turnbuckles must be attached to the differential pivot and rocker-bogie arm. We will accomplish this by drilling holes in the aluminum beam \textbf{S16} and attaching two 5-hole aluminum bars on each side.

Using a vice or clamp, firmly grab onto the 0.5 x 13.5 inch rod \textbf{S16B} with the end extending out around 2 inches from the edge of the vice/clamp. Mark the dimensions as shown in Figure \ref{diff pivot rod}. Carefully use a center drill to start the a center hole for these holes. It is important that the center hole is as centered as possible to prevent the bit from walking/slipping during drilling, which could result in the bit breaking. Then, use a \#23 (0.154 inch diameter) drill bit and drill all the way through both sides of the rod. This makes the \textbf{S16B'} part.

\begin{figure}[H]
\centering
  \includegraphics[width=1\linewidth]{"Pictures/Differential Pivot Cut".png}
	\caption{Drilling the Differential pivot rod}
	\label{diff pivot rod}
\end{figure}


Test the holes by taking the 5 Hole Aluminum Beams \textbf{S21} and screws \textbf{B7} and making sure that the screws go all the way through as shown in Figure \ref{test}. If they do not fit, you can file/drill the hole out until they do pass all the way through\footnote{The hole size on the Aluminum Beam \textbf{S21} is halfway between a \# 4 and \# 6 screw; we used the \# 6 screws and just used a hex key to drive the screw through the holes in beam, effectively tapping them as well.}.

\begin{figure}[H]
	\centering
	\includegraphics[width=0.6\textwidth]{"Pictures/Differential Align".PNG}
	\caption{Testing the differential pivot holes}
	\label{test}
\end{figure}

\noindent Flip the rod around and repeat the steps for the other side, making sure to align the holes' axes as much as possible such that the holes are all parallel to the previous set.


\noindent Next, take two of the 0.5x3 inch hollow rods \textbf{S18} and create the same set of holes as before, showed again in Figure \ref{dpv} (this time, drill holes on just one end of each of the rods). Test each of the sets of holes to make sure the 5-hole aluminum beams will attach to each of the rods. These will now be the part \textbf{S18B}.

\begin{figure}[H]
	\centering
	\includegraphics[width=0.6\textwidth]{"Pictures/Differential Standoff Cut".PNG}
	\caption{Drilling the smaller rods}
	\label{dpv}
\end{figure}

\newpage

\section{Mechanical Assembly}

\begin{table}[H]
    \centering
    \arrayrulecolor{lightgray}
    \sffamily\footnotesize
    \captionsetup{font={sf,bf}}
    \caption{Parts/Tools Necessary}
    \begin{tabular}{|N|Q|Q|I|N|Q|Q|I|}
        \hline
        \thead{Item} & \thead{Ref} & \thead{Qty} & \thead{Image} & \thead{Item} & \thead{Ref} & \thead{Qty} & \thead{Image} \\
        \hline
        Single Pattern Bracket & S8 & 1 & \partimg{../../../images/parts_list/S8.jpg} & \#6-32 x 1.25" Button Head Screw & B7 & 8 & \partimg{../../../images/parts_list/B7.png} \\ \hline
        0.5" Circular Clamping Hub & S13 & 1 & \partimg{../../../images/parts_list/S13.jpg} & \#4-40 x 1.25" Button Head Screw & B9 & 4 & \partimg{../../../images/parts_list/B9.png} \\ \hline
        0.5" Bottom Bore Clamp & S20 & 2 & \partimg{../../../images/parts_list/S20.jpg} & \#6-32 Locking Hex Nut & B11 & 8 & \partimg{../../../images/parts_list/B11.png} \\ \hline
        0.5" x 14" Aluminum Tube & S16B & 1 & \partimg{../../../images/parts_list/S16.png} & \#4-40 Locking Hex Nut & B12 & 4 & \partimg{../../../images/parts_list/B12.png} \\ \hline
        0.5" x 3" Aluminum Tube & S18B & 2 & \partimg{../../../images/parts_list/S18.jpg} & \#4-40 Washer & W2 & 24 & \partimg{../../../images/parts_list/W2.png} \\ \hline
        5 Hole Aluminum Beam & S21 & 8 & \partimg{../../../images/parts_list/S21.jpg} & Allen Key Set & & & \partimg{../../../images/parts_list/D2.jpeg} \\ \hline
        RC Turnbuckles & S32 & 2 & \partimg{../../../images/parts_list/S32.jpg} & Wrench Set & & & \partimg{../../../images/parts_list/D1.jpg} \\ \hline
        \#6-32 x 1/4" Button Head Screw & B1 & 8 & \partimg{../../../images/parts_list/B1.png} & & & & \\ \hline
    \end{tabular}
\end{table}

\begin{enumerate}
\item \textbf{Attach Clamping Hubs:} Use screws \textbf{B1} to attach the bottom tapped clamping hubs \textbf{S20} to the single pattern bracket \textbf{S8}. Then use screws \textbf{B1} to attach the 0.5 inch clamping hub \textbf{S13} to the bottom of the pattern bracket (Figure \ref{attaching clamping hub}).

\begin{figure}[H]
  \centering
  \begin{minipage}[b]{0.45\textwidth}
    \includegraphics[width=\textwidth]{"Pictures/Diff Step 1".PNG}
  \end{minipage}
  \hfill
  \begin{minipage}[b]{0.45\textwidth}
    \includegraphics[width=\textwidth]{"Pictures/Diff Step 2".PNG}
  \end{minipage}
  \caption{Attaching clamping hub}
  \label{attaching clamping hub}
\end{figure}

\item \textbf{Differential Pivot:} Attach the turnbuckle \textbf{S32} to the modified 13.5-inch aluminum rod \textbf{S16B} as shown using the 5-hole aluminum bars \textbf{S21}, washers \textbf{W2}, screws \textbf{B7} and \textbf{B9}, hex nuts \textbf{B11} and \textbf{B12}. The outermost screw is the \#4 screw, the others are \#6.

\begin{figure}[H]
  \centering
  \begin{minipage}[b]{0.45\textwidth}
    \includegraphics[width=\textwidth]{"Pictures/Diff Step 3".PNG}
  \end{minipage}
  \hfill
  \begin{minipage}[b]{0.45\textwidth}
    \includegraphics[width=\textwidth]{"Pictures/Diff Step 4".png}
  \end{minipage}
  \caption{Attaching the turnbuckle}
\end{figure}

\item \textbf{Differential Pivot continued:} Pass the 13.5inch rod through the clamping hub assembly, making sure to center it as much as possible. Then repeat step 2 for the other side of the differential pivot. If necessary, unscrew the turnbuckles (by twisting the middle) to insert the rods into place, then screw the turnbuckle back together.

\begin{figure}[H]
  \centering
  \begin{minipage}[b]{0.45\textwidth}
    \includegraphics[width=\textwidth]{"Pictures/Diff Step 5".PNG}
  \end{minipage}
  \hfill
  \begin{minipage}[b]{0.45\textwidth}
    \includegraphics[width=\textwidth]{"Pictures/Diff Step 6".png}
  \end{minipage}
  \caption{Attaching the top of the Differential Pivot}
\end{figure}

\item \textbf{Differential Pivot Vertical rods:} Repeat step 2 on each of the the 3 inch aluminum rods \textbf{S18B}. Finally, attach all the pieces of the turnbuckles together.  Your differential pivot is now complete.

\begin{figure}[H]
\centering
  \includegraphics[width=1\linewidth]{"Pictures/Diff Step 7".png}
	\caption{Final Differential Pivot Assembly}
\end{figure}


\end{enumerate}

\end{document}
